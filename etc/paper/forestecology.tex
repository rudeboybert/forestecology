% !TeX program = pdfLaTeX
\documentclass[12pt]{article}
\usepackage{amsmath}
\usepackage{graphicx,psfrag,epsf}
\usepackage{enumerate}
\usepackage{natbib}
\usepackage{textcomp}
\usepackage[hyphens]{url} % not crucial - just used below for the URL
\usepackage{hyperref}
\providecommand{\tightlist}{%
  \setlength{\itemsep}{0pt}\setlength{\parskip}{0pt}}

%\pdfminorversion=4
% NOTE: To produce blinded version, replace "0" with "1" below.
\newcommand{\blind}{0}

% DON'T change margins - should be 1 inch all around.
\addtolength{\oddsidemargin}{-.5in}%
\addtolength{\evensidemargin}{-.5in}%
\addtolength{\textwidth}{1in}%
\addtolength{\textheight}{1.3in}%
\addtolength{\topmargin}{-.8in}%

%% load any required packages here


\usepackage{color}
\usepackage{fancyvrb}
\newcommand{\VerbBar}{|}
\newcommand{\VERB}{\Verb[commandchars=\\\{\}]}
\DefineVerbatimEnvironment{Highlighting}{Verbatim}{commandchars=\\\{\}}
% Add ',fontsize=\small' for more characters per line
\usepackage{framed}
\definecolor{shadecolor}{RGB}{248,248,248}
\newenvironment{Shaded}{\begin{snugshade}}{\end{snugshade}}
\newcommand{\AlertTok}[1]{\textcolor[rgb]{0.94,0.16,0.16}{#1}}
\newcommand{\AnnotationTok}[1]{\textcolor[rgb]{0.56,0.35,0.01}{\textbf{\textit{#1}}}}
\newcommand{\AttributeTok}[1]{\textcolor[rgb]{0.77,0.63,0.00}{#1}}
\newcommand{\BaseNTok}[1]{\textcolor[rgb]{0.00,0.00,0.81}{#1}}
\newcommand{\BuiltInTok}[1]{#1}
\newcommand{\CharTok}[1]{\textcolor[rgb]{0.31,0.60,0.02}{#1}}
\newcommand{\CommentTok}[1]{\textcolor[rgb]{0.56,0.35,0.01}{\textit{#1}}}
\newcommand{\CommentVarTok}[1]{\textcolor[rgb]{0.56,0.35,0.01}{\textbf{\textit{#1}}}}
\newcommand{\ConstantTok}[1]{\textcolor[rgb]{0.00,0.00,0.00}{#1}}
\newcommand{\ControlFlowTok}[1]{\textcolor[rgb]{0.13,0.29,0.53}{\textbf{#1}}}
\newcommand{\DataTypeTok}[1]{\textcolor[rgb]{0.13,0.29,0.53}{#1}}
\newcommand{\DecValTok}[1]{\textcolor[rgb]{0.00,0.00,0.81}{#1}}
\newcommand{\DocumentationTok}[1]{\textcolor[rgb]{0.56,0.35,0.01}{\textbf{\textit{#1}}}}
\newcommand{\ErrorTok}[1]{\textcolor[rgb]{0.64,0.00,0.00}{\textbf{#1}}}
\newcommand{\ExtensionTok}[1]{#1}
\newcommand{\FloatTok}[1]{\textcolor[rgb]{0.00,0.00,0.81}{#1}}
\newcommand{\FunctionTok}[1]{\textcolor[rgb]{0.00,0.00,0.00}{#1}}
\newcommand{\ImportTok}[1]{#1}
\newcommand{\InformationTok}[1]{\textcolor[rgb]{0.56,0.35,0.01}{\textbf{\textit{#1}}}}
\newcommand{\KeywordTok}[1]{\textcolor[rgb]{0.13,0.29,0.53}{\textbf{#1}}}
\newcommand{\NormalTok}[1]{#1}
\newcommand{\OperatorTok}[1]{\textcolor[rgb]{0.81,0.36,0.00}{\textbf{#1}}}
\newcommand{\OtherTok}[1]{\textcolor[rgb]{0.56,0.35,0.01}{#1}}
\newcommand{\PreprocessorTok}[1]{\textcolor[rgb]{0.56,0.35,0.01}{\textit{#1}}}
\newcommand{\RegionMarkerTok}[1]{#1}
\newcommand{\SpecialCharTok}[1]{\textcolor[rgb]{0.00,0.00,0.00}{#1}}
\newcommand{\SpecialStringTok}[1]{\textcolor[rgb]{0.31,0.60,0.02}{#1}}
\newcommand{\StringTok}[1]{\textcolor[rgb]{0.31,0.60,0.02}{#1}}
\newcommand{\VariableTok}[1]{\textcolor[rgb]{0.00,0.00,0.00}{#1}}
\newcommand{\VerbatimStringTok}[1]{\textcolor[rgb]{0.31,0.60,0.02}{#1}}
\newcommand{\WarningTok}[1]{\textcolor[rgb]{0.56,0.35,0.01}{\textbf{\textit{#1}}}}

% Pandoc citation processing

\usepackage{xcolor, soul, xspace, float, subfig}

\begin{document}


\def\spacingset#1{\renewcommand{\baselinestretch}%
{#1}\small\normalsize} \spacingset{1}


%%%%%%%%%%%%%%%%%%%%%%%%%%%%%%%%%%%%%%%%%%%%%%%%%%%%%%%%%%%%%%%%%%%%%%%%%%%%%%

\if0\blind
{
  \title{\bf \texttt{forestecology} package for modeling interspecies competition
between trees}

  \author{
        Albert Y. Kim \thanks{Albert Y. Kim is Assistant Professor, Statistical \& Data Sciences,
Smith College, Northampton, MA 01063 (e-mail:
\href{mailto:akim04@smith.edu}{\nolinkurl{akim04@smith.edu}}). This work
was not supported by any grant. The authors thank numerous colleagues
and students for their support.} \\
    Program in Statistical \& Data Sciences, Smith College\\
     and \\     David Allen \\
    Biology Department, Middlebury College\\
      }
  \maketitle
} \fi

\if1\blind
{
  \bigskip
  \bigskip
  \bigskip
  \begin{center}
    {\LARGE\bf \texttt{forestecology} package for modeling interspecies competition
between trees}
  \end{center}
  \medskip
} \fi

\bigskip
\begin{abstract}
We provide a computational exercise suitable for early introduction in
an undergraduate statistics or data science course that allows students
to `play the whole game' of data science: performing both data
collection and data analysis. While many teaching resources exist for
data analysis, such resources are not as abundant for data collection
given the inherent difficulty of the task. Our proposed exercise centers
around student use of Google Calendar to collect data with the goal of
answering the question `How do I spend my time?' On the one hand, the
exercise involves answering a question with near universal appeal, but
on the other hand, the data collection mechanism is not beyond the reach
of a typical undergraduate student. A further benefit of the exercise is
that it provides an opportunity for discussions on ethical questions and
considerations that data providers and data analysts face in today's age
of large-scale internet-based data collection.
\end{abstract}

\noindent%
{\it Keywords:} forestecology, trees
\vfill

\newpage
\spacingset{1.45} % DON'T change the spacing!

\hypertarget{introduction}{%
\section{Introduction}\label{introduction}}

Repeat-censused forest plots offer excellent data to test neighborhood
models of tree competition \citet{allen_permutation_2020}
\citet{canham_neighborhood_2006} \citet{uriarte_spatially_2004}. Here we
describe an R package, \texttt{forestecology}, to do that. This package
implements the methods in \citet{allen_permutation_2020}. It provides: a
convenient way specify and fit models of tree growth based on
neighborhood competition; a spatial cross validation method to test and
compare model fits \citet{roberts_cross-validation_2017}; and an
ANOVA-like method to assess whether the competitor identity matters in
these models. The model is written to work with ForestGEO plot data
\citet{andersonteixeira_ctfs-forestgeo_2015}, but we envision that it
could easily be modified to work with data from other forest plots,
e.g.~the US Forest Service Forest Inventory and Analysis plots
\citet{smith_forest_2002}.

\hypertarget{example}{%
\section{Example}\label{example}}

We demonstrate the \texttt{forestecology} package's features on two data
sets, both based on inventory censuses of two sites from the Smithsonian
Institution's ForestGEO international network of 72 long‐term forest
dynamics research sites \citet{andersonteixeira_ctfs-forestgeo_2015}.
First, the Michigan Big Woods Forest Dynamics Plot located at the Edwin
S. George Reserve in Pinckney, MI, USA. The 23 ha plot is situated in
mature oak-hickory forest. The canopy is dominated by white oak (Quercus
alba), northern red oak (Quercus rubra), black oak (Quercus velutina),
shagbark hickory (Carya ovata) and pignut hickory (Carya glabra). The
most common understory tree is witch-hazel (Hamamelis virginiana)
\citet{allen_michigan_2020}. In the example below, we will preface any
data frames from this plot in with \texttt{bw\_}.

Second, the Smithsonian Conservation Biology Institute (SCBI) large
forest dynamics plot, located at the Smithsonian's National Zoo and
Conservation Biology Institute in Front Royal, VA, USA. The 25.6 ha (640
x 400 m) plot is located at the intersection of three of the major
physiographic provinces of the eastern US: the Blue Ridge, Ridge and
Valley, and Piedmont provinces and is adjacent to the northern end of
Shenandoah National Park. The forest type is typical mature secondary
eastern mixed deciduous forest, with a canopy dominated by tulip poplar
(Liriodendron tulipifera), oaks (Quercus spp.), and hickories (Carya
spp.), and an understory composed mainly of spicebush (Lindera benzoin),
paw-paw (Asimina triloba), American hornbeam (Carpinus caroliniana), and
witch hazel (Hamamelis virginiana) \citet{bourg_initial_2013}. In the
example below, we will preface any data frames from this plot in with
\texttt{scbi\_}.

We load all the necessary packages.

\begin{Shaded}
\begin{Highlighting}[]
\KeywordTok{library}\NormalTok{(forestecology)}

\CommentTok{# Load tidyverse packages:}
\KeywordTok{library}\NormalTok{(tidyverse)}
\KeywordTok{library}\NormalTok{(lubridate)}

\CommentTok{# Load spatial packages:}
\CommentTok{# devtools::install_github("rvalavi/blockCV")}
\KeywordTok{library}\NormalTok{(blockCV)}
\KeywordTok{library}\NormalTok{(sf)}
\KeywordTok{library}\NormalTok{(sfheaders)}

\CommentTok{# Load other packages:}
\KeywordTok{library}\NormalTok{(snakecase)}
\KeywordTok{library}\NormalTok{(yardstick)}
\end{Highlighting}
\end{Shaded}

\hypertarget{preprocess-census-data}{%
\subsection{Preprocess census data}\label{preprocess-census-data}}

We start by preprocessing the census data for both sites. While
ForestGEO data protocols ensure a high degree of standardization between
site, minor variations still exist
\citet{andersonteixeira_ctfs-forestgeo_2015}. While the Big Woods data
comes pre-loaded in the \texttt{forestecology} package, we load the SCBI
data as they are saved in .csv files in the SCBI-ForestGEO-Data
repository on GitHub
\citet{gonzalez-akre_scbi-forestgeoscbi-forestgeo-data_2020}. In both
cases, we load the census data as R as ``tibble'' data frames thereby
ensuring a standardized input/output format that can be used across all
\texttt{tidyverse} packages \citet{wickham_welcome_2019}.

Furthermore, we ensure that the different variables have the correct
names, types (\texttt{dbl}, \texttt{data}, \texttt{factor}).

\hypertarget{big-woods}{%
\subsubsection{Big Woods}\label{big-woods}}

We load census data from 2008 and 2014 saved in the package, then merge
species data (genus, species, linnean classification, family, etc).

\begin{Shaded}
\begin{Highlighting}[]
\KeywordTok{data}\NormalTok{(bw_census_}\DecValTok{2008}\NormalTok{, bw_census_}\DecValTok{2014}\NormalTok{, bw_species)}

\CommentTok{# Append additional species data}
\NormalTok{bw_census_}\DecValTok{2008}\NormalTok{ <-}\StringTok{ }\NormalTok{bw_census_}\DecValTok{2008} \OperatorTok
\StringTok{  }\KeywordTok{left_join}\NormalTok{(bw_species, }\DataTypeTok{by =} \StringTok{"sp"}\NormalTok{) }\OperatorTok
\StringTok{  }\KeywordTok{select}\NormalTok{(}\OperatorTok{-}\KeywordTok{c}\NormalTok{(genus, species, latin))}
\end{Highlighting}
\end{Shaded}

\hypertarget{scbi-data}{%
\subsubsection{SCBI}\label{scbi-data}}

We load census data from 2008 and 2014 from \texttt{.csv} files saved
from GitHub on November 20, 2020. Furthermore, we perform two additional
pre-processing steps. First, in order to speed up computation for
purposes of this example, we only consider a 9 ha subsection of the 25.6
ha of the SCBI site: \texttt{gx} from 0--300 instead of 0--400 and
\texttt{gy} from 300--600 instead of 0--640. Second, in order to
standardize comparisons between Big Woods and SCBI, we convert the units
of dbh from mm to cm. \footnote{A rule of thumb to ascertain if dbh is
  in mm or cm is to verify if the smallest non-zero and non-missing
  measurement is 1 or 10. If the former, then cm. If the later, then mm.
  This is because ForestGEO protocols state that only trees with dbh
  greater or equal to 1cm should be included in censuses. }

\begin{Shaded}
\begin{Highlighting}[]
\NormalTok{scbi_}\DecValTok{2013}\NormalTok{ <-}\StringTok{ }\KeywordTok{read_csv}\NormalTok{(}\StringTok{"scbi.stem2.csv"}\NormalTok{) }\OperatorTok
\StringTok{  }\KeywordTok{select}\NormalTok{(treeID, stemID, sp, ExactDate, gx, gy, dbh, codes, status) }\OperatorTok
\StringTok{  }\KeywordTok{mutate}\NormalTok{(}
    \DataTypeTok{date =}\NormalTok{ ExactDate,}
    \DataTypeTok{dbh =} \KeywordTok{as.numeric}\NormalTok{(dbh),}
    \DataTypeTok{date =} \KeywordTok{mdy}\NormalTok{(date)}
\NormalTok{  ) }\OperatorTok
\StringTok{  }\KeywordTok{filter}\NormalTok{(gx }\OperatorTok{<}\StringTok{ }\DecValTok{300}\NormalTok{, }\KeywordTok{between}\NormalTok{(gy, }\DecValTok{300}\NormalTok{, }\DecValTok{600}\NormalTok{)) }\OperatorTok\StringTok{ }
\StringTok{  }\KeywordTok{mutate}\NormalTok{(}\DataTypeTok{dbh =}\NormalTok{ dbh }\OperatorTok{/}\StringTok{ }\DecValTok{10}\NormalTok{)}

\NormalTok{scbi_}\DecValTok{2018}\NormalTok{ <-}\StringTok{ }\KeywordTok{read_csv}\NormalTok{(}\StringTok{"scbi.stem3.csv"}\NormalTok{) }\OperatorTok
\StringTok{  }\KeywordTok{select}\NormalTok{(treeID, stemID, sp, ExactDate, gx, gy, dbh, codes, status) }\OperatorTok
\StringTok{  }\KeywordTok{mutate}\NormalTok{(}
    \DataTypeTok{date =}\NormalTok{ ExactDate,}
    \DataTypeTok{dbh =} \KeywordTok{as.numeric}\NormalTok{(dbh),}
    \DataTypeTok{date =} \KeywordTok{mdy}\NormalTok{(date)}
\NormalTok{  ) }\OperatorTok
\StringTok{  }\KeywordTok{filter}\NormalTok{(gx }\OperatorTok{<}\StringTok{ }\DecValTok{300}\NormalTok{, }\KeywordTok{between}\NormalTok{(gy, }\DecValTok{300}\NormalTok{, }\DecValTok{600}\NormalTok{)) }\OperatorTok\StringTok{ }
\StringTok{  }\KeywordTok{mutate}\NormalTok{(}\DataTypeTok{dbh =}\NormalTok{ dbh }\OperatorTok{/}\StringTok{ }\DecValTok{10}\NormalTok{)}
\end{Highlighting}
\end{Shaded}

\hypertarget{compute-annual-growth}{%
\subsection{Compute annual growth}\label{compute-annual-growth}}

For each plot we then compute average annual growth between the two
censuses using the \texttt{compute\_growth()} function. This function
takes the two census data frames as well as a character indicating which
variable in both data frames uniquely identifies each stem. This
function returns a single data frame that includes a numerical variable
\texttt{growth} reflecting the average annual dbh growth (in cm) of all
trees alive at both time points. Furthermore, variables that (in theory)
remain unchanged between censuses appear only once, such as location
variables \texttt{gx} and \texttt{gy}; as well as species-related
variables. Variables that should change between censuses are suffixed
with \texttt{1} and \texttt{2} indicating the earlier and later
censuses, such as \texttt{dbh1/dbh2} and \texttt{codes1/codes2}. Here
the resulting data frames are named with some variation of
\texttt{growth\_df}.

After computing the average annual growth for each tree, we ensure to
convert all variables denote species from type character to factors;
this is to ensure that issues of rare species being accounted for in
both training and test sets in our upcoming cross-validation step (see
Section REF)

\hypertarget{big-woods-1}{%
\subsubsection{Big Woods}\label{big-woods-1}}

In the case of Big Woods data, we first remove all trees that were
re-sprouts in the later (2014) census. Additionally, we have included
three classification of tree species: \texttt{species}, \texttt{family},
and \texttt{trait\_group}. DESCRIBE THESE

\begin{Shaded}
\begin{Highlighting}[]
\NormalTok{bw_census_}\DecValTok{2014}\NormalTok{ <-}\StringTok{ }\NormalTok{bw_census_}\DecValTok{2014} \OperatorTok
\StringTok{  }\KeywordTok{filter}\NormalTok{(}\OperatorTok{!}\KeywordTok{str_detect}\NormalTok{(codes, }\StringTok{"R"}\NormalTok{))}

\NormalTok{bw_growth_df <-}
\StringTok{  }\KeywordTok{compute_growth}\NormalTok{(bw_census_}\DecValTok{2008}\NormalTok{, bw_census_}\DecValTok{2014}\NormalTok{, }\DataTypeTok{id =} \StringTok{"treeID"}\NormalTok{) }\OperatorTok
\StringTok{  }\CommentTok{# Convert all variables denoting species to factors}
\StringTok{  }\KeywordTok{mutate}\NormalTok{(}
    \DataTypeTok{sp =}\NormalTok{ sp }\OperatorTok\StringTok{ }\KeywordTok{to_any_case}\NormalTok{() }\OperatorTok\StringTok{ }\KeywordTok{as.factor}\NormalTok{(),}
    \DataTypeTok{species =}\NormalTok{ sp,}
    \DataTypeTok{family =} \KeywordTok{as.factor}\NormalTok{(family),}
    \DataTypeTok{trait_group =} \KeywordTok{as.factor}\NormalTok{(trait_group)}
\NormalTok{  ) }\OperatorTok
\StringTok{  }\CommentTok{# Drop unnecessary variables}
\StringTok{  }\KeywordTok{select}\NormalTok{(}\OperatorTok{-}\NormalTok{stemID)}
\end{Highlighting}
\end{Shaded}

\hypertarget{scbi}{%
\subsubsection{SCBI}\label{scbi}}

\begin{Shaded}
\begin{Highlighting}[]
\NormalTok{scbi_growth_df <-}
\StringTok{  }\KeywordTok{compute_growth}\NormalTok{(scbi_}\DecValTok{2013}\NormalTok{, scbi_}\DecValTok{2018}\NormalTok{, }\StringTok{"stemID"}\NormalTok{) }\OperatorTok
\StringTok{  }\CommentTok{# Convert all variables denoting species to factors}
\StringTok{  }\KeywordTok{mutate}\NormalTok{(}\DataTypeTok{sp =} \KeywordTok{as.factor}\NormalTok{(sp))}
\end{Highlighting}
\end{Shaded}

\hypertarget{comparison}{%
\subsubsection{Comparison}\label{comparison}}

Figure \ref{fig:growth-histogram} displays histograms comparing the
distribution of average annual growth at both sites. Observe that
average annual growth appears higher at the Big Woods site.

\begin{Shaded}
\begin{Highlighting}[]
\CommentTok{# Both Big Woods & SCBI}
\KeywordTok{bind_rows}\NormalTok{(}
\NormalTok{  bw_growth_df }\OperatorTok\StringTok{ }\KeywordTok{st_drop_geometry}\NormalTok{() }\OperatorTok\StringTok{ }\KeywordTok{select}\NormalTok{(growth) }\OperatorTok\StringTok{ }\KeywordTok{mutate}\NormalTok{(}\DataTypeTok{Site =} \StringTok{"Big Woods"}\NormalTok{),}
\NormalTok{  scbi_growth_df }\OperatorTok\StringTok{ }\KeywordTok{st_drop_geometry}\NormalTok{() }\OperatorTok\StringTok{ }\KeywordTok{select}\NormalTok{(growth) }\OperatorTok\StringTok{ }\KeywordTok{mutate}\NormalTok{(}\DataTypeTok{Site =} \StringTok{"SCBI"}\NormalTok{)}
\NormalTok{) }\OperatorTok\StringTok{ }
\StringTok{  }\KeywordTok{ggplot}\NormalTok{(}\KeywordTok{aes}\NormalTok{(}\DataTypeTok{x =}\NormalTok{ growth, }\DataTypeTok{y =}\NormalTok{ ..density.., }\DataTypeTok{fill =}\NormalTok{ Site)) }\OperatorTok{+}
\StringTok{  }\KeywordTok{geom_histogram}\NormalTok{(}\DataTypeTok{alpha =} \FloatTok{0.5}\NormalTok{, }\DataTypeTok{position =} \StringTok{"identity"}\NormalTok{, }\DataTypeTok{binwidth =} \FloatTok{0.05}\NormalTok{) }\OperatorTok{+}
\StringTok{  }\KeywordTok{labs}\NormalTok{(}\DataTypeTok{x =} \StringTok{"Average annual growth in dbh (cm per yr)"}\NormalTok{) }\OperatorTok{+}
\StringTok{  }\KeywordTok{coord_cartesian}\NormalTok{(}\DataTypeTok{xlim =} \KeywordTok{c}\NormalTok{(}\OperatorTok{-}\FloatTok{0.5}\NormalTok{, }\DecValTok{1}\NormalTok{))}
\end{Highlighting}
\end{Shaded}

\begin{figure}

{\centering \includegraphics[width=1\linewidth]{Figures/growth-histogram-1} 

}

\caption{Distribution of average annual growth in DBH for both sites.}\label{fig:growth-histogram}
\end{figure}

\hypertarget{spatial-information}{%
\subsection{Add spatial information}\label{spatial-information}}

We now encode spatial information to the \texttt{growth\_df} data
frames. First, in order to control for study region edge effects, we add
``buffers'' to the periphery of the study region (cite Waller?). Our
model of interspecific competition relies on a spatial definition of who
the competitor trees are for focal trees of interest. Since certain
explanatory variables such as biomass are cumulative, we must ensure
that all trees being modeled are not biased to have different neighbor
structures. This is a particular concern for trees at the boundary of
study regions, which will not have the same number of neighbors as trees
in the internal part of the study region.

Second, our ultimate method for model assessment will rely on estimates
of model error as generated by cross-validation. Conventional
cross-validation schemes assign observations to folds by resampling
individual observations at random. However, underlying this scheme is an
assumption that the observations are independent. In the case of forest
census data, observations exhibit spatial autocorrelation, and thus this
dependence must be incorporated in our resampling scheme in spatial
cross-validation \citet{roberts2017} \citet{pohjankukka2017}. We will
therefore associate portions of the study region to spatial folds.

To these two ends, we define two constants, both of which are in the
same units as the \texttt{gx} and \texttt{gy} variables (most often
meters).

\begin{Shaded}
\begin{Highlighting}[]
\NormalTok{max_dist <-}\StringTok{ }\FloatTok{7.5}
\NormalTok{cv_fold_size <-}\StringTok{ }\DecValTok{100}
\end{Highlighting}
\end{Shaded}

The first constant is \texttt{max\_dist} which defines the maximum
distance for a tree's competitive neighborhood. Trees within this
distance of each other are assumed to compete while those farther than
this distance apart do not. Put differently, all trees within
\texttt{max\_dist} of a focal tree will be considered its competitors
(see below). Other studies have estimated the value of
\texttt{max\_dist}; we use an average of estimated values
\citet{canham_neighborhood_2004}, \citet{uriarte_spatially_2004},
\citet{tatsumi2013}, \citet{canham_neighborhood_2006}.

Furthermore, \texttt{max\_dist} will define the size of all buffers
considered, which will be encoded as a binary variable \texttt{buffer}
as computed by the \texttt{add\_buffer\_variable()} function. This
function takes as input the main \texttt{growth\_df} data frame, the
\texttt{size} of the buffer which we set as \texttt{max\_dist}, and the
boundary of the study region encoded as a simple features polygon
\citet{pebesma_simple_2018}. DESCRIBE SF PACKAGE. In the Big Woods
example below we will use a pre-loaded simple features polygon while for
the SCBI example we present example code on how to manually construct
one.

The second constant is \texttt{cv\_fold\_size} which defines the length
and width of the spatial folds (note that for now the spatial folds are
restricted be squares). We will then use this constant to associate each
observed tree to one of \(k\) folds in the respective study region. In
the Big Woods example below we will use the \texttt{blockCV} R package
that has implemented spatial cross-validation while for the SCBI we will
do this manually \citet{valavi2019}.

\hypertarget{big-woods-2}{%
\subsubsection{Big Woods}\label{big-woods-2}}

First, we indicate which trees are part of the buffer. This necessitates
information about the study region boundary. In this case, we use a
\texttt{sf\_polygon} object \texttt{bw\_study\_region} which comes
pre-loaded in the \texttt{forestecology} packages. After loading
\texttt{bw\_study\_region}, we illustrate the results of the
\texttt{add\_buffer\_variable()} function in Figure
\ref{fig:bw-define-buffer}. Trees on the periphery denote with lighter
colors are part of the buffer and will not be considered as ``focal''
trees of interest going forward; they will only be considered as
competitor trees.

\begin{Shaded}
\begin{Highlighting}[]
\KeywordTok{data}\NormalTok{(bw_study_region)}

\NormalTok{bw_growth_df <-}\StringTok{ }\NormalTok{bw_growth_df }\OperatorTok
\StringTok{  }\KeywordTok{add_buffer_variable}\NormalTok{(}\DataTypeTok{direction =} \StringTok{"in"}\NormalTok{, }\DataTypeTok{size =}\NormalTok{ max_dist, }\DataTypeTok{region =}\NormalTok{ bw_study_region)}

\KeywordTok{ggplot}\NormalTok{() }\OperatorTok{+}
\StringTok{  }\KeywordTok{geom_sf}\NormalTok{(}\DataTypeTok{data =}\NormalTok{ bw_growth_df }\OperatorTok\StringTok{ }\KeywordTok{sample_frac}\NormalTok{(}\FloatTok{0.2}\NormalTok{), }\KeywordTok{aes}\NormalTok{(}\DataTypeTok{col =}\NormalTok{ buffer), }\DataTypeTok{size =} \FloatTok{0.5}\NormalTok{)}
\end{Highlighting}
\end{Shaded}

\begin{figure}

{\centering \includegraphics[width=1\linewidth]{Figures/bw-define-buffer-1} 

}

\caption{Buffer region for Big Woods study region.}\label{fig:bw-define-buffer}
\end{figure}

Second, we associate each tree to spatial cross validation folds. In
this case, we use the \texttt{spatialBlock()} function from the
\texttt{blockCV} package to define the spatial grid which

THIS IS A MESS. We use the \citet{valavi_blockcv_2019}, whose elements
will act as the folds in our leave-one-out (by ``one'' we mean ``one
grid block'') cross-validation scheme. The upshot here is we add
\texttt{foldID} to \texttt{growth\_df} which identifies which fold each
individual is in, and the creation of a \texttt{cv\_grid\_sf} object
which gives the geometry of the cross validation grid.

\begin{Shaded}
\begin{Highlighting}[]
\KeywordTok{set.seed}\NormalTok{(}\DecValTok{76}\NormalTok{)}
\NormalTok{bw_spatialBlock <-}\StringTok{ }\KeywordTok{spatialBlock}\NormalTok{(}
  \DataTypeTok{speciesData =}\NormalTok{ bw_growth_df, }\DataTypeTok{theRange =}\NormalTok{ cv_fold_size,}
  \DataTypeTok{k =} \DecValTok{28}\NormalTok{, }\DataTypeTok{xOffset =} \FloatTok{0.5}\NormalTok{, }\DataTypeTok{yOffset =} \DecValTok{0}\NormalTok{,}
  \DataTypeTok{verbose =} \OtherTok{FALSE}
\NormalTok{)}

\CommentTok{# Add foldID to each tree}
\NormalTok{bw_growth_df <-}\StringTok{ }\NormalTok{bw_growth_df }\OperatorTok
\StringTok{  }\KeywordTok{mutate}\NormalTok{(}\DataTypeTok{foldID =}\NormalTok{ bw_spatialBlock}\OperatorTok{$}\NormalTok{foldID)}

\CommentTok{# Visualize grid. Why does fold 19 repeat?}
\NormalTok{bw_spatialBlock}\OperatorTok{$}\NormalTok{plots }\OperatorTok{+}
\StringTok{  }\KeywordTok{geom_sf}\NormalTok{(}\DataTypeTok{data =}\NormalTok{ bw_growth_df }\OperatorTok\StringTok{ }\KeywordTok{sample_frac}\NormalTok{(}\FloatTok{0.2}\NormalTok{), }\KeywordTok{aes}\NormalTok{(}\DataTypeTok{col =} \KeywordTok{factor}\NormalTok{(foldID)), }\DataTypeTok{size =} \FloatTok{0.1}\NormalTok{)}

\CommentTok{# Remove empty folds}
\NormalTok{bw_growth_df <-}\StringTok{ }\NormalTok{bw_growth_df }\OperatorTok
\StringTok{  }\KeywordTok{filter}\NormalTok{(}\OperatorTok{!}\NormalTok{foldID }\OperatorTok\StringTok{ }\KeywordTok{c}\NormalTok{(}\DecValTok{19}\NormalTok{, }\DecValTok{23}\NormalTok{, }\DecValTok{21}\NormalTok{, }\DecValTok{17}\NormalTok{, }\DecValTok{8}\NormalTok{, }\DecValTok{19}\NormalTok{)) }\OperatorTok
\StringTok{  }\KeywordTok{mutate}\NormalTok{(}\DataTypeTok{foldID =} \KeywordTok{as.character}\NormalTok{(foldID))}
\end{Highlighting}
\end{Shaded}

\begin{center}\includegraphics[width=1\linewidth]{Figures/bw-define-cv-folds-1} \includegraphics[width=1\linewidth]{Figures/bw-define-cv-folds-2} \end{center}

Separately, we save the spatial cross-validation grid as an
\texttt{sf\_polygon} object \texttt{bw\_cv\_grid}

\begin{Shaded}
\begin{Highlighting}[]
\NormalTok{bw_cv_grid <-}\StringTok{ }\NormalTok{bw_spatialBlock}\OperatorTok{$}\NormalTok{blocks }\OperatorTok
\StringTok{  }\KeywordTok{st_as_sf}\NormalTok{()}
\end{Highlighting}
\end{Shaded}

\hypertarget{scbi-1}{%
\subsubsection{SCBI}\label{scbi-1}}

First, we indicate which trees are part of the buffer. In this case
however we manually define the study region boundary based on the
subregion we defined in Section \ref{scbi-data} and create an
\texttt{sf\_polygon} object using the \texttt{sf\_polygon()} function
from the \texttt{sfheaders} package. Figure \ref{fig:scbi-define-buffer}
displays the resulting buffer trees.

\begin{Shaded}
\begin{Highlighting}[]
\NormalTok{scbi_study_region <-}\StringTok{ }\KeywordTok{tibble}\NormalTok{(}
  \DataTypeTok{x =} \KeywordTok{c}\NormalTok{(}\DecValTok{0}\NormalTok{, }\DecValTok{300}\NormalTok{, }\DecValTok{300}\NormalTok{, }\DecValTok{0}\NormalTok{, }\DecValTok{0}\NormalTok{),}
  \DataTypeTok{y =} \KeywordTok{c}\NormalTok{(}\DecValTok{300}\NormalTok{, }\DecValTok{300}\NormalTok{, }\DecValTok{600}\NormalTok{, }\DecValTok{600}\NormalTok{, }\DecValTok{300}\NormalTok{)}
\NormalTok{) }\OperatorTok
\StringTok{  }\KeywordTok{sf_polygon}\NormalTok{()}

\NormalTok{scbi_growth_df <-}\StringTok{ }\NormalTok{scbi_growth_df }\OperatorTok
\StringTok{  }\KeywordTok{add_buffer_variable}\NormalTok{(}\DataTypeTok{direction =} \StringTok{"in"}\NormalTok{, }\DataTypeTok{size =}\NormalTok{ max_dist, }\DataTypeTok{region =}\NormalTok{ scbi_study_region)}

\KeywordTok{ggplot}\NormalTok{() }\OperatorTok{+}
\StringTok{  }\KeywordTok{geom_sf}\NormalTok{(}\DataTypeTok{data =}\NormalTok{ scbi_growth_df, }\KeywordTok{aes}\NormalTok{(}\DataTypeTok{col =}\NormalTok{ buffer), }\DataTypeTok{size =} \FloatTok{0.5}\NormalTok{)}
\end{Highlighting}
\end{Shaded}

\begin{figure}

{\centering \includegraphics[width=0.5\linewidth]{Figures/scbi-define-buffer-1} 

}

\caption{Buffer region for SCBI study region.}\label{fig:scbi-define-buffer}
\end{figure}

Second, we associate each tree to spatial cross validation folds. In
this case we manually define a spatial crossvaliation grid. Figure
\ref{fig:scbi-define-cv-folds} displays the resulting cross-validation
folds along with the buffer from Figure \ref{fig:scbi-define-buffer}.

Here we manually define the spatial cross-validation grid as an
\texttt{sf\_polygon} object \texttt{scbi\_cv\_grid}

\begin{Shaded}
\begin{Highlighting}[]
\NormalTok{fold1 <-}\StringTok{ }\KeywordTok{rbind}\NormalTok{(}\KeywordTok{c}\NormalTok{(}\DecValTok{0}\NormalTok{, }\DecValTok{300}\NormalTok{), }\KeywordTok{c}\NormalTok{(}\DecValTok{150}\NormalTok{, }\DecValTok{300}\NormalTok{), }\KeywordTok{c}\NormalTok{(}\DecValTok{150}\NormalTok{, }\DecValTok{600}\NormalTok{), }\KeywordTok{c}\NormalTok{(}\DecValTok{0}\NormalTok{, }\DecValTok{600}\NormalTok{), }\KeywordTok{c}\NormalTok{(}\DecValTok{0}\NormalTok{, }\DecValTok{300}\NormalTok{)) }\OperatorTok
\StringTok{  }\KeywordTok{sf_polygon}\NormalTok{() }\OperatorTok\StringTok{ }
\StringTok{  }\KeywordTok{mutate}\NormalTok{(}\DataTypeTok{folds =} \DecValTok{1}\NormalTok{)}
\NormalTok{fold2 <-}\StringTok{ }\KeywordTok{rbind}\NormalTok{(}\KeywordTok{c}\NormalTok{(}\DecValTok{150}\NormalTok{, }\DecValTok{300}\NormalTok{), }\KeywordTok{c}\NormalTok{(}\DecValTok{300}\NormalTok{, }\DecValTok{300}\NormalTok{), }\KeywordTok{c}\NormalTok{(}\DecValTok{300}\NormalTok{, }\DecValTok{600}\NormalTok{), }\KeywordTok{c}\NormalTok{(}\DecValTok{150}\NormalTok{, }\DecValTok{600}\NormalTok{), }\KeywordTok{c}\NormalTok{(}\DecValTok{150}\NormalTok{, }\DecValTok{300}\NormalTok{)) }\OperatorTok
\StringTok{  }\KeywordTok{sf_polygon}\NormalTok{() }\OperatorTok\StringTok{ }
\StringTok{  }\KeywordTok{mutate}\NormalTok{(}\DataTypeTok{folds =} \DecValTok{2}\NormalTok{)}
\NormalTok{scbi_cv_grid <-}\StringTok{ }\KeywordTok{bind_rows}\NormalTok{(fold1, fold2)}
\end{Highlighting}
\end{Shaded}

\begin{Shaded}
\begin{Highlighting}[]
\NormalTok{scbi_spatialBlock <-}\StringTok{ }\KeywordTok{spatialBlock}\NormalTok{(}
  \DataTypeTok{speciesData =}\NormalTok{ scbi_growth_df,}
  \DataTypeTok{k =} \DecValTok{2}\NormalTok{,}
  \DataTypeTok{verbose =} \OtherTok{FALSE}\NormalTok{,}
  \DataTypeTok{showBlocks =} \OtherTok{FALSE}\NormalTok{,}
  \CommentTok{# Note new arguments:}
  \DataTypeTok{selection =} \StringTok{"systematic"}\NormalTok{,}
  \DataTypeTok{blocks =}\NormalTok{ scbi_cv_grid}
\NormalTok{)}

\CommentTok{# Add foldID to each tree}
\NormalTok{scbi_growth_df <-}\StringTok{ }\NormalTok{scbi_growth_df }\OperatorTok
\StringTok{  }\KeywordTok{mutate}\NormalTok{(}\DataTypeTok{foldID =}\NormalTok{ scbi_spatialBlock}\OperatorTok{$}\NormalTok{foldID)}

\KeywordTok{ggplot}\NormalTok{() }\OperatorTok{+}
\StringTok{  }\KeywordTok{geom_sf_text}\NormalTok{(}\DataTypeTok{data =}\NormalTok{ scbi_growth_df, }\KeywordTok{aes}\NormalTok{(}\DataTypeTok{label =}\NormalTok{ foldID, }\DataTypeTok{col =}\NormalTok{ buffer)) }\OperatorTok{+}
\StringTok{  }\KeywordTok{geom_sf}\NormalTok{(}\DataTypeTok{data =}\NormalTok{ scbi_cv_grid, }\DataTypeTok{fill =} \StringTok{"transparent"}\NormalTok{)}
\end{Highlighting}
\end{Shaded}

\begin{figure}

{\centering \includegraphics[width=0.5\linewidth]{Figures/scbi-define-cv-folds-1} 

}

\caption{Buffer region for SCBI study region.}\label{fig:scbi-define-cv-folds}
\end{figure}

\hypertarget{define-focal-versus-competitor-trees}{%
\subsection{Define focal versus competitor
trees}\label{define-focal-versus-competitor-trees}}

Next we define \texttt{focal\_vs\_comp} data frames which connects each
focal tree in the \texttt{growth\_df} data frames to the trees in its
competitive neighborhood range as defined by the \texttt{max\_dist}
constant. So for example, if \texttt{growth\_df} consisted of two focal
trees with two and three neighbors with \texttt{max\_dist} respectively,
\texttt{focal\_vs\_comp} would be a data frame of 5 rows connecting each
focal tree to it's competitors. The \texttt{create\_focal\_vs\_comp()}
function makes this connection taking as inputs the \texttt{growth\_df}
data frame; the \texttt{max\_dist} constant defining competitive range;
\texttt{cv\_grid\_sf}, giving the cross validation grid; and the
\texttt{id} variable.

\hypertarget{big-woods-3}{%
\subsubsection{Big Woods}\label{big-woods-3}}

\begin{Shaded}
\begin{Highlighting}[]
\NormalTok{focal_vs_comp_bw <-}\StringTok{ }\NormalTok{bw_growth_df }\OperatorTok
\StringTok{  }\KeywordTok{create_focal_vs_comp}\NormalTok{(max_dist, }\DataTypeTok{cv_grid_sf =}\NormalTok{ bw_cv_grid, }\DataTypeTok{id =} \StringTok{"treeID"}\NormalTok{)}
\end{Highlighting}
\end{Shaded}

TODO: Figure out how to show this data frame's contents.

\begin{Shaded}
\begin{Highlighting}[]
\KeywordTok{head}\NormalTok{(focal_vs_comp_bw)}
\CommentTok{## # A tibble: 6 x 10}
\CommentTok{##   focal_ID focal_sp   dbh foldID                  geometry growth}
\CommentTok{##      <dbl> <fct>    <dbl> <chr>                    <POINT>  <dbl>}
\CommentTok{## 1        1 white_o~  41.2 12                   (8.7 107.5)  0.404}
\CommentTok{## 2        1 white_o~  41.2 12                   (8.7 107.5)  0.404}
\CommentTok{## 3        1 white_o~  41.2 12                   (8.7 107.5)  0.404}
\CommentTok{## 4        1 white_o~  41.2 12                   (8.7 107.5)  0.404}
\CommentTok{## 5        1 white_o~  41.2 12                   (8.7 107.5)  0.404}
\CommentTok{## 6        1 white_o~  41.2 12                   (8.7 107.5)  0.404}
\CommentTok{## # ... with 4 more variables: comp_ID <dbl>, dist <dbl>,}
\CommentTok{## #   comp_sp <fct>, comp_basal_area <dbl>}
\end{Highlighting}
\end{Shaded}

\hypertarget{scbi-2}{%
\subsubsection{SCBI}\label{scbi-2}}

\begin{Shaded}
\begin{Highlighting}[]
\NormalTok{focal_vs_comp_scbi <-}\StringTok{ }\NormalTok{scbi_growth_df }\OperatorTok
\StringTok{  }\KeywordTok{create_focal_vs_comp}\NormalTok{(max_dist, }\DataTypeTok{cv_grid_sf =}\NormalTok{ scbi_cv_grid, }\DataTypeTok{id =} \StringTok{"stemID"}\NormalTok{)}
\end{Highlighting}
\end{Shaded}

TODO: Figure out how to show this data frame's contents.

\begin{Shaded}
\begin{Highlighting}[]
\KeywordTok{head}\NormalTok{(focal_vs_comp_scbi)}
\CommentTok{## # A tibble: 6 x 10}
\CommentTok{##   focal_ID focal_sp   dbh foldID                  geometry growth}
\CommentTok{##      <dbl> <fct>    <dbl>  <int>                   <POINT>  <dbl>}
\CommentTok{## 1        4 nysy      13.6      1              (14.2 428.5)  0.103}
\CommentTok{## 2        4 nysy      13.6      1              (14.2 428.5)  0.103}
\CommentTok{## 3        4 nysy      13.6      1              (14.2 428.5)  0.103}
\CommentTok{## 4        4 nysy      13.6      1              (14.2 428.5)  0.103}
\CommentTok{## 5        4 nysy      13.6      1              (14.2 428.5)  0.103}
\CommentTok{## 6        4 nysy      13.6      1              (14.2 428.5)  0.103}
\CommentTok{## # ... with 4 more variables: comp_ID <dbl>, dist <dbl>,}
\CommentTok{## #   comp_sp <fct>, comp_basal_area <dbl>}
\end{Highlighting}
\end{Shaded}

\hypertarget{model-fit-predict}{%
\subsection{Fit model and make predictions}\label{model-fit-predict}}

Next we fit the following linear model to the dbh of each focal tree.
Let \(i = 1, \ldots, n_j\) index all \(n_j\) trees of
\texttt{focal\textquotesingle{}\textquotesingle{}\ species\ group\ \$j\$;\ let\ \$j\ =\ 1,\ \textbackslash{}ldots,\ J\$\ index\ all\ \$J\$\ focal\ species\ groups;\ and\ let\ \$k\ =\ 1,\ \textbackslash{}ldots,\ K\$\ index\ all\ \$K\$}competitor''
species groups. We modeled the growth in diameter per year \(y_{ij}\)
(in centimeters per year) of the \(i^{th}\) tree of focal species group
\(j\) as a linear model \(f\) of the following covariates
\(\vec{x}_{ij}\)

\[
\newcommand{\dbh}{\text{DBH}}
\newcommand{\biomass}{\text{biomass}}
\newcommand{\BA}{\text{BA}}
y_{ij} = f(\vec{x}_{ij}) + \epsilon_{ij} = \beta_{0,j} + \beta_{\dbh,j} \cdot \dbh_{ij} + \sum_{k=1}^{K} \lambda_{jk} \cdot \BA_{ijk} + \epsilon_{ij}
\]

We estimate the model's parameters using Bayesian linear regression
implemented in the \texttt{fit\_bayesian\_model()} function. TODO:
define all parameters

For this linear model's case, there exists a closed form solution as
described here. As such, the \texttt{fit\_bayesian\_model()} function
using matrix algebra to obtain all parameter estimates, rather than
computationally expensive Monte Carlo approximations. The inputs to this
function are a \texttt{focal\_vs\_comp} data frame,
\texttt{prior\_param} a list of priors, and a boolean flag
\texttt{run\_shuffle} on whether or not to run competitor-species
identity permutations which we will demonstrate below on the Michigan
Big Woods data. This function returns the posterior means of all
parameters.

Using these posterior means, we then use the posterior predictive
distribution to obtain fitted/predicted values \(\widehat{y}\) of the
dbh for each focal tree using the \texttt{predict\_bayesian\_model()}.
These \(\widehat{y}\) can then be compared to the observed \(y\) dbh's
to compute the root mean-square error, a measure of a model's predictive
error which has the same units as the observed data \(y\).

\hypertarget{big-woods-4}{%
\subsubsection{Big Woods}\label{big-woods-4}}

For the Michigan Big Woods data we present two use cases of the model
fitting and prediction scheme. The first use case is the simplest where
we assess the fit of the model using root mean squared error. The second
use case then answers the question of whether species competitor
identity matters using permutation test.

For the first use case, we fit the linear model specified in Equation
XXX to our data frame of type \texttt{focal\_vs\_comp}. This
input/outputs of the \texttt{fit\_bayesian\_model()} function are lists
of the prior/posterior means of parameters of the linear regression
specified in XXX. Generally speaking, there are two classes of
regression parameters: \(\beta\) main effects and \(\lambda\)
competitive effects. In the upcoming Section
\ref{viz-posterior-distributions}, we will present code visualizing this
posterior distributions.

\begin{Shaded}
\begin{Highlighting}[]
\NormalTok{posterior_param_bw <-}\StringTok{ }\NormalTok{focal_vs_comp_bw }\OperatorTok
\StringTok{  }\KeywordTok{fit_bayesian_model}\NormalTok{(}\DataTypeTok{prior_param =} \OtherTok{NULL}\NormalTok{)}
\end{Highlighting}
\end{Shaded}

This output of posterior parameters for the specified competition model
are then used along with the posterior predictive distribution encoded
in \texttt{predict\_bayesian\_model()} to return predicted growths for
each individual tree. We join these predicted growths to the original
growth data frame.

\begin{Shaded}
\begin{Highlighting}[]
\NormalTok{predictions <-}\StringTok{ }\NormalTok{focal_vs_comp_bw }\OperatorTok
\StringTok{  }\KeywordTok{predict_bayesian_model}\NormalTok{(}\DataTypeTok{posterior_param =}\NormalTok{ posterior_param_bw) }\OperatorTok
\StringTok{  }\KeywordTok{right_join}\NormalTok{(bw_growth_df, }\DataTypeTok{by =} \KeywordTok{c}\NormalTok{(}\StringTok{"focal_ID"}\NormalTok{ =}\StringTok{ "treeID"}\NormalTok{))}
\end{Highlighting}
\end{Shaded}

We then use the \texttt{rmse()} function from the \texttt{yardstick}
package to obtain the root mean squared error of the observed versus
fitted values of growth.

\begin{Shaded}
\begin{Highlighting}[]
\NormalTok{predictions }\OperatorTok
\StringTok{  }\NormalTok{yardstick}\OperatorTok{::}\KeywordTok{rmse}\NormalTok{(}\DataTypeTok{truth =}\NormalTok{ growth, }\DataTypeTok{estimate =}\NormalTok{ growth_hat) }\OperatorTok
\StringTok{  }\KeywordTok{pull}\NormalTok{(.estimate)}
\CommentTok{## [1] 0.148145}
\end{Highlighting}
\end{Shaded}

The second use case is near identical to the first, but with a small
change in the code to test whether the identity of the competitor
matters. By adding a \texttt{run\_shuffle\ =\ TRUE} argument to
\texttt{fit\_bayesian\_model()}, for each focal tree its competitor
trees' species identity will be ``shuffled'' randomly much like in a
permutation test. By shuffling these species labels we are effectively
fitting the model under a null model that competitor species identity
does not matter. If the ``shuffled'' RMSE's are consistently lower than
the unshuffled RMSE corresponding to the observed data, then we have
evidence to suggest that competitor identity matters to competitive
interactions.

\begin{Shaded}
\begin{Highlighting}[]
\NormalTok{posterior_param_bw_shuffle <-}\StringTok{ }\NormalTok{focal_vs_comp_bw }\OperatorTok
\StringTok{  }\KeywordTok{fit_bayesian_model}\NormalTok{(}\DataTypeTok{prior_param =} \OtherTok{NULL}\NormalTok{, }\DataTypeTok{run_shuffle =} \OtherTok{TRUE}\NormalTok{)}
\end{Highlighting}
\end{Shaded}

\begin{Shaded}
\begin{Highlighting}[]
\NormalTok{predictions_shuffle <-}\StringTok{ }\NormalTok{focal_vs_comp_bw }\OperatorTok
\StringTok{  }\KeywordTok{predict_bayesian_model}\NormalTok{(}\DataTypeTok{posterior_param =}\NormalTok{ posterior_param_bw_shuffle) }\OperatorTok
\StringTok{  }\KeywordTok{right_join}\NormalTok{(bw_growth_df, }\DataTypeTok{by =} \KeywordTok{c}\NormalTok{(}\StringTok{"focal_ID"}\NormalTok{ =}\StringTok{ "treeID"}\NormalTok{))}
\end{Highlighting}
\end{Shaded}

\begin{Shaded}
\begin{Highlighting}[]
\NormalTok{predictions_shuffle }\OperatorTok
\StringTok{  }\KeywordTok{rmse}\NormalTok{(}\DataTypeTok{truth =}\NormalTok{ growth, }\DataTypeTok{estimate =}\NormalTok{ growth_hat) }\OperatorTok
\StringTok{  }\KeywordTok{pull}\NormalTok{(.estimate)}
\CommentTok{## [1] 0.1505383}
\end{Highlighting}
\end{Shaded}

The RMSE is fact lower for the non-shuffled version, indicative of a
better model fit. This gives support for the idea that competitor
identity does matter for competitive interactions. In
\citet{allen_permutation_2020} we run this shuffle a large number of
times to construct a full permutation distribution to show that this
difference is robust to resampling variation.

\hypertarget{scbi-3}{%
\subsubsection{SCBI}\label{scbi-3}}

In the case of the SCBI data, we once again perform the same model
fitting and computing of fitted growths as with the Big Woods data, but
this time we map the residuals of the observed minus fitted values to
look for spatial patterns.

\begin{Shaded}
\begin{Highlighting}[]
\NormalTok{posterior_param_scbi <-}\StringTok{ }\NormalTok{focal_vs_comp_scbi }\OperatorTok
\StringTok{  }\KeywordTok{fit_bayesian_model}\NormalTok{(}\DataTypeTok{prior_param =} \OtherTok{NULL}\NormalTok{, }\DataTypeTok{run_shuffle =} \OtherTok{FALSE}\NormalTok{)}
\end{Highlighting}
\end{Shaded}

\begin{Shaded}
\begin{Highlighting}[]
\NormalTok{scbi_growth_df_noCV <-}\StringTok{ }\NormalTok{focal_vs_comp_scbi }\OperatorTok
\StringTok{  }\KeywordTok{predict_bayesian_model}\NormalTok{(}\DataTypeTok{posterior_param =}\NormalTok{ posterior_param_scbi) }\OperatorTok
\StringTok{  }\KeywordTok{right_join}\NormalTok{(scbi_growth_df, }\DataTypeTok{by =} \KeywordTok{c}\NormalTok{(}\StringTok{"focal_ID"}\NormalTok{ =}\StringTok{ "stemID"}\NormalTok{))}
\end{Highlighting}
\end{Shaded}

\begin{Shaded}
\begin{Highlighting}[]
\NormalTok{scbi_growth_df_noCV }\OperatorTok
\StringTok{  }\KeywordTok{rmse}\NormalTok{(}\DataTypeTok{truth =}\NormalTok{ growth, }\DataTypeTok{estimate =}\NormalTok{ growth_hat) }\OperatorTok
\StringTok{  }\KeywordTok{pull}\NormalTok{(.estimate)}
\CommentTok{## [1] 0.1280644}
\end{Highlighting}
\end{Shaded}

In Figures \ref{fig:scbi-model-residuals} and
\ref{fig:scbi-model-residuals-2} we present the residuals.

\begin{Shaded}
\begin{Highlighting}[]
\KeywordTok{ggplot}\NormalTok{(scbi_growth_df_noCV, }\KeywordTok{aes}\NormalTok{(}\DataTypeTok{x =}\NormalTok{ growth, }\DataTypeTok{y =}\NormalTok{ growth_hat)) }\OperatorTok{+}
\StringTok{  }\KeywordTok{geom_point}\NormalTok{(}\DataTypeTok{size =} \FloatTok{0.5}\NormalTok{, }\DataTypeTok{color =} \KeywordTok{rgb}\NormalTok{(}\DecValTok{0}\NormalTok{, }\DecValTok{0}\NormalTok{, }\DecValTok{0}\NormalTok{, }\FloatTok{0.25}\NormalTok{)) }\OperatorTok{+}
\StringTok{  }\KeywordTok{stat_smooth}\NormalTok{(}\DataTypeTok{method =} \StringTok{"lm"}\NormalTok{) }\OperatorTok{+}
\StringTok{  }\KeywordTok{geom_abline}\NormalTok{(}\DataTypeTok{slope =} \DecValTok{1}\NormalTok{, }\DataTypeTok{intercept =} \DecValTok{0}\NormalTok{) }\OperatorTok{+}
\StringTok{  }\KeywordTok{coord_fixed}\NormalTok{() }\OperatorTok{+}
\StringTok{  }\KeywordTok{labs}\NormalTok{(}
    \DataTypeTok{x =} \StringTok{"Observed growth in dbh"}\NormalTok{, }\DataTypeTok{y =} \StringTok{"Predicted growth in dbh"}\NormalTok{,}
    \DataTypeTok{title =} \StringTok{"Predicted vs Observed Growth"}
\NormalTok{  )}
\end{Highlighting}
\end{Shaded}

\begin{figure}

{\centering \includegraphics[width=1\linewidth]{Figures/scbi-model-residuals-1} 

}

\caption{Spatial distribution of residuals for model applied to SCBI data.}\label{fig:scbi-model-residuals}
\end{figure}

\begin{Shaded}
\begin{Highlighting}[]
\NormalTok{scbi_growth_df_noCV }\OperatorTok
\StringTok{  }\KeywordTok{st_as_sf}\NormalTok{() }\OperatorTok
\StringTok{  }\CommentTok{# }\AlertTok{TODO}\CommentTok{: Need to investigate missingness}
\StringTok{  }\KeywordTok{filter}\NormalTok{(}\OperatorTok{!}\KeywordTok{is.na}\NormalTok{(growth_hat)) }\OperatorTok
\StringTok{  }\KeywordTok{mutate}\NormalTok{(}
    \DataTypeTok{error =}\NormalTok{ growth }\OperatorTok{-}\StringTok{ }\NormalTok{growth_hat,}
    \DataTypeTok{error_bin =} \KeywordTok{cut_number}\NormalTok{(error, }\DataTypeTok{n =} \DecValTok{5}\NormalTok{),}
    \DataTypeTok{error_compress =} \KeywordTok{ifelse}\NormalTok{(error }\OperatorTok{<}\StringTok{ }\FloatTok{-0.75}\NormalTok{, }\FloatTok{-0.75}\NormalTok{, }\KeywordTok{ifelse}\NormalTok{(error }\OperatorTok{>}\StringTok{ }\FloatTok{0.75}\NormalTok{, }\FloatTok{0.75}\NormalTok{, error))}
\NormalTok{  ) }\OperatorTok
\StringTok{  }\KeywordTok{ggplot}\NormalTok{() }\OperatorTok{+}
\StringTok{  }\KeywordTok{geom_sf}\NormalTok{(}\KeywordTok{aes}\NormalTok{(}\DataTypeTok{col =}\NormalTok{ error_compress), }\DataTypeTok{size =} \DecValTok{1}\NormalTok{) }\OperatorTok{+}
\StringTok{  }\KeywordTok{theme_bw}\NormalTok{() }\OperatorTok{+}
\StringTok{  }\KeywordTok{scale_color_gradient2}\NormalTok{(}
    \DataTypeTok{low =} \StringTok{"#ef8a62"}\NormalTok{, }\DataTypeTok{mid =} \StringTok{"#f7f7f7"}\NormalTok{, }\DataTypeTok{high =} \StringTok{"#67a9cf"}\NormalTok{,}
    \DataTypeTok{name =} \KeywordTok{expression}\NormalTok{(}\KeywordTok{paste}\NormalTok{(}\StringTok{"Residual (cm "}\NormalTok{, y}\OperatorTok{^}\NormalTok{\{}\OperatorTok{-}\DecValTok{1}\NormalTok{\}, }\StringTok{")"}\NormalTok{)),}
    \DataTypeTok{breaks =} \KeywordTok{seq}\NormalTok{(}\DataTypeTok{from =} \FloatTok{-0.75}\NormalTok{, }\DataTypeTok{to =} \FloatTok{0.75}\NormalTok{, }\DataTypeTok{by =} \FloatTok{0.25}\NormalTok{),}
    \DataTypeTok{labels =} \KeywordTok{c}\NormalTok{(}\StringTok{"< -0.75"}\NormalTok{, }\StringTok{"-0.5"}\NormalTok{, }\StringTok{"0.25"}\NormalTok{, }\StringTok{"0"}\NormalTok{, }\StringTok{"0.25"}\NormalTok{, }\StringTok{"0.5"}\NormalTok{, }\StringTok{"> 0.75"}\NormalTok{)}
\NormalTok{  ) }\OperatorTok{+}
\StringTok{  }\KeywordTok{labs}\NormalTok{(}\DataTypeTok{x =} \StringTok{"Meter"}\NormalTok{, }\DataTypeTok{y =} \StringTok{"Meter"}\NormalTok{)}
\end{Highlighting}
\end{Shaded}

\begin{figure}

{\centering \includegraphics[width=1\linewidth]{Figures/scbi-model-residuals-2-1} 

}

\caption{Spatial distribution of residuals for model applied to SCBI data part 2.}\label{fig:scbi-model-residuals-2}
\end{figure}

\hypertarget{run-spatial-cross-validation}{%
\subsection{Run spatial
cross-validation}\label{run-spatial-cross-validation}}

The model fits and predictions in Section \ref{model-fit-predict} all
suffer from a common failing: they use the same data to both fit the
model and to assess the model's performance using the RMSE. As argued by
\citet{roberts_cross-validation_2017}, this can lead to overly
optimistic assessments of model quality as the models can be overfit, in
particular in situations where spatial-autocorrelation is present. To
mitigate the effects of such overfitting, we use a spatially block
cross-validation algorithm implemented in the \texttt{run\_cv()}. This
function at its core uses the same model fitting implemented in the
\texttt{fit\_bayesian\_model()} function, however trains the model on
\(k-1\) spatial folds of the train and returns fitted values for the
test data. Recall that the spatial blocking scheme wass encoded in
Section \ref{spatial-information}.

\hypertarget{big-woods-5}{%
\subsubsection{Big Woods}\label{big-woods-5}}

Applying this spatially cross-validated model fit yields an RMSE is
higher than that when the model is fit without cross validation. In
other words, our model fits in \ref{model-fit-predict} were overly
optimistic in the model's fitting power, whereas a cross-validated
results yield an estimate that is closer to the truth. See
\citet{allen_permutation_2020} for more discussion of this.

\begin{Shaded}
\begin{Highlighting}[]
\NormalTok{cv_bw <-}\StringTok{ }\NormalTok{focal_vs_comp_bw }\OperatorTok
\StringTok{  }\KeywordTok{run_cv}\NormalTok{(}\DataTypeTok{max_dist =}\NormalTok{ max_dist, }\DataTypeTok{cv_grid =}\NormalTok{ bw_cv_grid) }\OperatorTok
\StringTok{  }\KeywordTok{right_join}\NormalTok{(bw_growth_df, }\DataTypeTok{by =} \KeywordTok{c}\NormalTok{(}\StringTok{"focal_ID"}\NormalTok{ =}\StringTok{ "treeID"}\NormalTok{))}

\NormalTok{cv_bw }\OperatorTok
\StringTok{  }\KeywordTok{rmse}\NormalTok{(}\DataTypeTok{truth =}\NormalTok{ growth, }\DataTypeTok{estimate =}\NormalTok{ growth_hat) }\OperatorTok
\StringTok{  }\KeywordTok{pull}\NormalTok{(.estimate)}
\CommentTok{## [1] 0.1533511}
\end{Highlighting}
\end{Shaded}

\hypertarget{scbi-4}{%
\subsubsection{SCBI}\label{scbi-4}}

Observe once again that this RMSE is much higher than that for the above
SCBI model fit without cross-validation.

\begin{Shaded}
\begin{Highlighting}[]
\NormalTok{cv_scbi <-}\StringTok{ }\NormalTok{focal_vs_comp_scbi }\OperatorTok
\StringTok{  }\KeywordTok{run_cv}\NormalTok{(}\DataTypeTok{max_dist =}\NormalTok{ max_dist, }\DataTypeTok{cv_grid =}\NormalTok{ scbi_cv_grid) }\OperatorTok
\StringTok{  }\KeywordTok{right_join}\NormalTok{(scbi_growth_df, }\DataTypeTok{by =} \KeywordTok{c}\NormalTok{(}\StringTok{"focal_ID"}\NormalTok{ =}\StringTok{ "treeID"}\NormalTok{))}

\NormalTok{cv_scbi }\OperatorTok
\StringTok{  }\KeywordTok{rmse}\NormalTok{(}\DataTypeTok{truth =}\NormalTok{ growth, }\DataTypeTok{estimate =}\NormalTok{ growth_hat) }\OperatorTok
\StringTok{  }\KeywordTok{pull}\NormalTok{(.estimate)}
\CommentTok{## [1] 0.1494775}
\end{Highlighting}
\end{Shaded}

\hypertarget{viz-posterior-distributions}{%
\subsection{Visualize posterior
distributions}\label{viz-posterior-distributions}}

Lastly, we return to the model fits from Section \ref{model-fit-predict}
and present tools to visually explore the posterior distributions of all
parameters in our model. There are two main groups of parameters to
consider. The \(\beta\) coefficients tell us about how fast each species
grows and how this depends on DBH while the full matrix of \(\lambda\)
values describe the competitive effects between pairs of species. There
is a rich literature on this matrix (cite).

DO WE NEED TO DESCRIBE MECHANICS? Because of the structure of the
\texttt{bw\_fit\_model} object we cannot simply draw these curves based
on the posterior distribution. \texttt{bw\_fit\_model()} gives the
parameters \emph{compared} to a baseline. This is not of direct
interest. So to display these parameters, as we care about them, we have
to sample from the baseline distribution and from the comparison one to
get the posterior distribution of interest.

\hypertarget{big-woods-6}{%
\subsubsection{Big Woods}\label{big-woods-6}}

Here we re-run the model fit to the Big Woods data from Section
\ref{model-fit-predict}, but this time use ``family'' as the group for
comparison which has. This makes the posterior distributions easier to
follow. Also, surprisingly, grouping by family performed just as well as
grouping by species \citet{allen_permutation_2020}. First we re-run
\texttt{create\_focal\_vs\_comp()} and \texttt{fit\_bayesian\_model()}
with no permutation shuffling with the grouping variable as family.

\begin{Shaded}
\begin{Highlighting}[]
\NormalTok{focal_vs_comp_bw <-}\StringTok{ }\NormalTok{bw_growth_df }\OperatorTok
\StringTok{  }\KeywordTok{mutate}\NormalTok{(}\DataTypeTok{sp =}\NormalTok{ family) }\OperatorTok
\StringTok{  }\KeywordTok{create_focal_vs_comp}\NormalTok{(}\DataTypeTok{max_dist =}\NormalTok{ max_dist, }\DataTypeTok{cv_grid_sf =}\NormalTok{ bw_cv_grid, }\DataTypeTok{id =} \StringTok{"treeID"}\NormalTok{)}

\NormalTok{posterior_param_bw <-}\StringTok{ }\NormalTok{focal_vs_comp_bw }\OperatorTok
\StringTok{  }\KeywordTok{fit_bayesian_model}\NormalTok{(}\DataTypeTok{prior_param =} \OtherTok{NULL}\NormalTok{, }\DataTypeTok{run_shuffle =} \OtherTok{FALSE}\NormalTok{)}
\end{Highlighting}
\end{Shaded}

Now the posterior parameter outputs of \texttt{fit\_bayesian\_model()}
are passed to \texttt{plot\_posterior\_parameters()} to generate
visualizations of the posterior parameters. These visualizations are
displayed in Figure 5 of \citet{allen_permutation_2020}. For simplicity
we only plot a subset of the species families.

\begin{Shaded}
\begin{Highlighting}[]
\NormalTok{posterior_plots <-}\StringTok{ }\KeywordTok{plot_posterior_parameters}\NormalTok{(}
  \DataTypeTok{posterior_param =}\NormalTok{ posterior_param_bw,}
  \DataTypeTok{sp_to_plot =} \KeywordTok{c}\NormalTok{(}\StringTok{"cornaceae"}\NormalTok{, }\StringTok{"fagaceae"}\NormalTok{, }\StringTok{"hamamelidaceae"}\NormalTok{, }\StringTok{"juglandaceae"}\NormalTok{, }
                 \StringTok{"lauraceae"}\NormalTok{, }\StringTok{"rosaceae"}\NormalTok{, }\StringTok{"sapindaceae"}\NormalTok{, }\StringTok{"ulmaceae"}\NormalTok{)}
\NormalTok{)}
\end{Highlighting}
\end{Shaded}

The output is a list with three plots stored. Figure
\ref{fig:bw-posterior-viz-beta0} The element \texttt{beta\_0} gives the
baseline growth intercept \(\beta_0\), i.e., how fast an individual of
each group grows independent of DBH).

\begin{Shaded}
\begin{Highlighting}[]
\NormalTok{posterior_plots[[}\StringTok{"beta_0"}\NormalTok{]]}
\end{Highlighting}
\end{Shaded}

\begin{figure}

{\centering \includegraphics[width=0.5\linewidth]{Figures/bw-posterior-viz-beta0-1} 

}

\caption{Posterior distribution of beta0.}\label{fig:bw-posterior-viz-beta0}
\end{figure}

Figure \ref{fig:bw-posterior-viz-beta-dbh} Next \texttt{beta\_dbh} gives
the slope for DBH slope \(\beta_{dbh,i}\) for each group.

\begin{Shaded}
\begin{Highlighting}[]
\NormalTok{posterior_plots[[}\StringTok{"beta_dbh"}\NormalTok{]]}
\end{Highlighting}
\end{Shaded}

\begin{figure}

{\centering \includegraphics[width=1\linewidth]{Figures/bw-posterior-viz-beta-dbh-1} 

}

\caption{Posterior distribution of betadbh.}\label{fig:bw-posterior-viz-beta-dbh}
\end{figure}

Finally Figure \ref{fig:bw-posterior-viz-lambda} \texttt{lambda} gives
the competition coefficients \(\lambda\).

\begin{Shaded}
\begin{Highlighting}[]
\NormalTok{posterior_plots[[}\StringTok{"lambda"}\NormalTok{]]}
\end{Highlighting}
\end{Shaded}

\begin{figure}

{\centering \includegraphics[width=1\linewidth]{Figures/bw-posterior-viz-lambda-1} 

}

\caption{Posterior distribution of lambda.}\label{fig:bw-posterior-viz-lambda}
\end{figure}

\hypertarget{scbi-5}{%
\subsubsection{SCBI}\label{scbi-5}}

We revisit the posterior parameters for the SCBI from Section
\{model-fit-predict\}, but this time only focus on the \(\lambda\)
competition coefficients.

\begin{Shaded}
\begin{Highlighting}[]
\NormalTok{posterior_plots_scbi <-}\StringTok{ }\KeywordTok{plot_posterior_parameters}\NormalTok{(}
  \DataTypeTok{posterior_param =}\NormalTok{ posterior_param_scbi,}
  \DataTypeTok{sp_to_plot =} \KeywordTok{c}\NormalTok{(}\StringTok{"quru"}\NormalTok{, }\StringTok{"litu"}\NormalTok{, }\StringTok{"cagl"}\NormalTok{, }\StringTok{"cato"}\NormalTok{)}
\NormalTok{)}
\end{Highlighting}
\end{Shaded}

\begin{Shaded}
\begin{Highlighting}[]
\NormalTok{posterior_plots_scbi[[}\StringTok{"lambda"}\NormalTok{]]}
\end{Highlighting}
\end{Shaded}

\begin{figure}

{\centering \includegraphics[width=1\linewidth]{Figures/scbi-posterior-viz-lambda-1} 

}

\caption{Posterior distribution of lambda.}\label{fig:scbi-posterior-viz-lambda}
\end{figure}

Add explanation here.

HEY BERT PICK IT UP HERE

\hypertarget{conclusions}{%
\section{Conclusions}\label{conclusions}}

\hypertarget{acknowledgments}{%
\section{Acknowledgments}\label{acknowledgments}}

\bibliographystyle{agsm}
\bibliography{paper.bib}

\end{document}
