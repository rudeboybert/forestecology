% !TeX program = pdfLaTeX
\documentclass[12pt]{article}
\usepackage{amsmath}
\usepackage{graphicx,psfrag,epsf}
\usepackage{enumerate}
\usepackage{natbib}
\usepackage{textcomp}
\usepackage[hyphens]{url} % not crucial - just used below for the URL
\usepackage{hyperref}
\providecommand{\tightlist}{%
  \setlength{\itemsep}{0pt}\setlength{\parskip}{0pt}}

%\pdfminorversion=4
% NOTE: To produce blinded version, replace "0" with "1" below.
\newcommand{\blind}{0}

% DON'T change margins - should be 1 inch all around.
\addtolength{\oddsidemargin}{-.5in}%
\addtolength{\evensidemargin}{-.5in}%
\addtolength{\textwidth}{1in}%
\addtolength{\textheight}{1.3in}%
\addtolength{\topmargin}{-.8in}%

%% load any required packages here


\usepackage{color}
\usepackage{fancyvrb}
\newcommand{\VerbBar}{|}
\newcommand{\VERB}{\Verb[commandchars=\\\{\}]}
\DefineVerbatimEnvironment{Highlighting}{Verbatim}{commandchars=\\\{\}}
% Add ',fontsize=\small' for more characters per line
\usepackage{framed}
\definecolor{shadecolor}{RGB}{248,248,248}
\newenvironment{Shaded}{\begin{snugshade}}{\end{snugshade}}
\newcommand{\AlertTok}[1]{\textcolor[rgb]{0.94,0.16,0.16}{#1}}
\newcommand{\AnnotationTok}[1]{\textcolor[rgb]{0.56,0.35,0.01}{\textbf{\textit{#1}}}}
\newcommand{\AttributeTok}[1]{\textcolor[rgb]{0.77,0.63,0.00}{#1}}
\newcommand{\BaseNTok}[1]{\textcolor[rgb]{0.00,0.00,0.81}{#1}}
\newcommand{\BuiltInTok}[1]{#1}
\newcommand{\CharTok}[1]{\textcolor[rgb]{0.31,0.60,0.02}{#1}}
\newcommand{\CommentTok}[1]{\textcolor[rgb]{0.56,0.35,0.01}{\textit{#1}}}
\newcommand{\CommentVarTok}[1]{\textcolor[rgb]{0.56,0.35,0.01}{\textbf{\textit{#1}}}}
\newcommand{\ConstantTok}[1]{\textcolor[rgb]{0.00,0.00,0.00}{#1}}
\newcommand{\ControlFlowTok}[1]{\textcolor[rgb]{0.13,0.29,0.53}{\textbf{#1}}}
\newcommand{\DataTypeTok}[1]{\textcolor[rgb]{0.13,0.29,0.53}{#1}}
\newcommand{\DecValTok}[1]{\textcolor[rgb]{0.00,0.00,0.81}{#1}}
\newcommand{\DocumentationTok}[1]{\textcolor[rgb]{0.56,0.35,0.01}{\textbf{\textit{#1}}}}
\newcommand{\ErrorTok}[1]{\textcolor[rgb]{0.64,0.00,0.00}{\textbf{#1}}}
\newcommand{\ExtensionTok}[1]{#1}
\newcommand{\FloatTok}[1]{\textcolor[rgb]{0.00,0.00,0.81}{#1}}
\newcommand{\FunctionTok}[1]{\textcolor[rgb]{0.00,0.00,0.00}{#1}}
\newcommand{\ImportTok}[1]{#1}
\newcommand{\InformationTok}[1]{\textcolor[rgb]{0.56,0.35,0.01}{\textbf{\textit{#1}}}}
\newcommand{\KeywordTok}[1]{\textcolor[rgb]{0.13,0.29,0.53}{\textbf{#1}}}
\newcommand{\NormalTok}[1]{#1}
\newcommand{\OperatorTok}[1]{\textcolor[rgb]{0.81,0.36,0.00}{\textbf{#1}}}
\newcommand{\OtherTok}[1]{\textcolor[rgb]{0.56,0.35,0.01}{#1}}
\newcommand{\PreprocessorTok}[1]{\textcolor[rgb]{0.56,0.35,0.01}{\textit{#1}}}
\newcommand{\RegionMarkerTok}[1]{#1}
\newcommand{\SpecialCharTok}[1]{\textcolor[rgb]{0.00,0.00,0.00}{#1}}
\newcommand{\SpecialStringTok}[1]{\textcolor[rgb]{0.31,0.60,0.02}{#1}}
\newcommand{\StringTok}[1]{\textcolor[rgb]{0.31,0.60,0.02}{#1}}
\newcommand{\VariableTok}[1]{\textcolor[rgb]{0.00,0.00,0.00}{#1}}
\newcommand{\VerbatimStringTok}[1]{\textcolor[rgb]{0.31,0.60,0.02}{#1}}
\newcommand{\WarningTok}[1]{\textcolor[rgb]{0.56,0.35,0.01}{\textbf{\textit{#1}}}}

% Pandoc citation processing

\usepackage{xcolor, soul, xspace, float, subfig, lineno, setspace, fancyhdr}
\linenumbers
\pagestyle{fancy}
\fancyhead[LO,LE]{forestecology R package}
\renewcommand{\sectionmark}[1]{\markright{#1}{}}

\begin{document}


\def\spacingset#1{\renewcommand{\baselinestretch}%
{#1}\small\normalsize} \spacingset{1}


%%%%%%%%%%%%%%%%%%%%%%%%%%%%%%%%%%%%%%%%%%%%%%%%%%%%%%%%%%%%%%%%%%%%%%%%%%%%%%

\if0\blind
{
  \title{\bf The forestecology R package for fitting and assessing neighborhood
models of the effect of interspecific competition on the growth of trees}

  \author{
        Albert Y. Kim \thanks{Assistant Professor, Statistical \& Data Sciences, Smith College,
Northampton, MA 01063 (e-mail:
\href{mailto:akim04@smith.edu}{\nolinkurl{akim04@smith.edu}}).} \\
    Program in Statistical \& Data Sciences, Smith College\\
     and \\     David N. Allen \\
    Biology Department, Middlebury College\\
     and \\     Simon P. Couch \\
    Mathematics Department, Reed College\\
      }
  \maketitle
} \fi

\if1\blind
{
  \bigskip
  \bigskip
  \bigskip
  \begin{center}
    {\LARGE\bf The forestecology R package for fitting and assessing neighborhood
models of the effect of interspecific competition on the growth of trees}
  \end{center}
  \medskip
} \fi

\bigskip
\begin{abstract}
1. Neighborhood competition models are powerful tools to measure the
effect of interspecific competition. Statistical methods to ease the
application of these models are currently lacking.\\
2. We present the \texttt{forestecology} package providing methods to i)
specify neighborhood competition models, ii) evalulate the effect of
competitor species identity using permutation tests, and iii) measure
model performance using spatial cross-validation. Following
\citet{allen_permutation_2020}, we implement a Bayesian linear
regression neighborhood competition model.\\
3. We demonstrate the package's functionality using data from the
Smithsonian Conservation Biology Institute's large forest dynamics plot,
part of the ForestGEO global network of reseach sites. Given ForestGEO's
data collection protocols and data formatting standards, the package was
designed with cross-site compatibility in mind. We highlight the
importance of spatial cross-validation when interpreting model
results.\\
4. The package features i) \texttt{tidyverse}-like structure whereby
verb-named functions can be modularly ``piped'' in sequence, ii)
functions with standardized inputs/outputs of simple features
\texttt{sf} package class, and iii) an S3 object-oriented implementation
of the Bayesian linear regression model. These three facts allow for
clear articulation of all the steps in the sequence of analysis and easy
wrangling and visualization of the geospatial forestry data.
Furthermore, while the package only has Bayesian linear regression
implemented, the package was designed with extensibility to other
methods in mind.
\end{abstract}

\noindent%
{\it Keywords:} forest ecology, interspecific competition, neigbhorhood competition,
tree growth, R, ForestGEO, spatial cross-validation
\vfill

\newpage
\spacingset{1.45} % DON'T change the spacing!

\doublespacing

\hypertarget{introduction}{%
\section{Introduction}\label{introduction}}

Repeat-censused forest plots offer excellent opportunities to test
neighborhood models of the effect of competition on the growth of trees
(\citet{canham_neighborhood_2004}). Neighborhood models of competition
have been used to: test whether the species identity of a competitor
matters (\citet{uriarte_spatially_2004}); measure species-specific
competition coefficients (\citet{das_effect_2012}
\citet{tatsumi_estimating_2016}); test competing models to see what
structures competitive interactions, e.g.~traits or phylogeny
(\citet{allen_permutation_2020}; \citet{uriarte_trait_2010}); and inform
selective logging practices (\citet{canham_neighborhood_2006}). Although
these are well-described methods, few methods are currently available
for easy application. Here we address this in an R package. We largely
follow the methods presented in \citet{allen_permutation_2020}. The
package is written to model stem radial growth between two censuses
based on neighborhood competition.

\citet{allen_permutation_2020} considers the following model: Let
\(i = 1, \ldots, n_j\) index all \(n_j\) trees of ``focal'' species
group \(j\); let \(j = 1, \ldots, J\) index all \(J\) focal species
groups; and let \(k = 1, \ldots, K\) index all \(K\) ``competitor''
species groups. We model the average annual growth in diameter at breast
height (DBH) \(y_{ij}\) (in centimeters per year) of the \(i^{th}\) tree
of focal species group \(j\) as a linear model \(f\) of the covariates
\(\vec{x}_{ij}\)

\begin{equation}
\label{eq:model}
y_{ij} = f(\vec{x}_{ij}) + \epsilon_{ij} = \beta_{0,j} + \beta_{\text{dbh},j} \cdot \text{dbh}_{ij} + \sum_{k=1}^{K} \lambda_{jk} \cdot \text{BA}_{ijk} + \epsilon_{ij}
\end{equation}

where \(\beta_{0,j}\) is the diameter-independent growth rate for group
\(j\); \(\text{dbh}_{ij}\) is the DBH the focal tree at the earlier
census; \(\beta_{\text{dbh},j}\) is the amount of the growth rate
changed depending on diameter for group \(j\); \(\text{BA}_{ijk}\) is
the sum of the basal area of all trees of competitor species group
\(k\); \(\lambda_{jk}\) is the change in growth for individuals of group
\(j\) from nearby competitors of group \(k\); and \(\epsilon_{ij}\) is a
random error term distributed \(\text{Normal}(0, \sigma^2)\). They
estimate all parameters via Bayesian linear regression while exploiting
Normal/Inverse Gamma conjugacy to derive closed-form solutions to all
posterior distributions\footnote{See S1 Appendix of
  \citet{allen_permutation_2020}, available at
  \url{https://doi.org/10.1371/journal.pone.0229930.s004}}. These
closed-form solutions for the posterior distributions are in contrast to
approximations of all posteriors via computationally expensive Markov
Chain Monte Carlo algorithms.

In order to evaluate whether competitor species identity matters,
\citet{allen_permutation_2020} run a permutation test where under the
null hypothesis the species identity of all competitors of a focal tree
can be permuted/shuffled:

\begin{eqnarray}
\label{eq:permutation-hypothesis-test}
&&H_0: \lambda_{jk} = \lambda_{j} \mbox{ for all } k = 1, \ldots, K\\
\text{vs.}&&H_A: \text{at least one } \lambda_{jk} \mbox{ is different}
\end{eqnarray}

where the null hypothesis \(H_0\) reflects a hypothesis of no species
grouping-specific effects of competition while the alternative
hypothesis \(H_A\) reflects a hypothesis of species grouping-specific
effects of competition. Furthermore, in order to account for the spatial
autocorrelation inherent to forest data in their estimates of
out-of-sample model error, \citet{allen_permutation_2020} use spatial
cross-validation. Estimates of model error that do not account for this
spatial dependency tend to underestimate the true model error
\citep{roberts_cross-validation_2017}.

We introduce the \texttt{forestecology} R package providing methods and
data for forest ecology model fitting and assessment, available on CRAN
(\url{https://cran.r-project.org/web/packages/forestecology/index.html})
and on GitHub (\url{https://github.com/rudeboybert/forestecology}). The
package implements all aspects of the model in Equation \ref{eq:model}:
model fitting and generating fitted/predicted values, evaluating the
effect of competitor species identity using permutation tests, and
evaluating model performance using spatial cross-validation.

The package designed with ``tidy'' design principles in mind
\citep{wickham_welcome_2019}. Much like many of the \texttt{tidyverse}
component packages, \texttt{forestecology} is designed with verb-named
functions that can be modularly composed in sequence using the pipe
\texttt{\%\textgreater{}\%} operator \citep{bache_pipe_2020}. As we
articulate in Section \ref{casestudy}, these functions delineate the key
steps in our analysis sequence. Furthermore, the inputs and outputs of
nearly all of our functions use the same ``simple features for R'' data
structures as implemented in the \texttt{sf} package for standardized
and \texttt{tidyverse}-friendly support for spatial vector data
\citep{pebesma_simple_2018}

Currently the package only implements the Bayesian linear regression
model of tree growth based on neighborhood competition detailed in
Equation \ref{eq:model}. As we demonstrate in Section
\ref{model-fit-predict} however, the fitting of this model is
self-contained in a single function \texttt{comp\_bayes\_lm()}. This
function returns an object of S3 class type \texttt{comp\_bayes\_lm}
with generic methods implemented to print, make predictions using, and
plot all results. Therefore the package can be modularly extended to fit
other models as long as they are coded into a function similar to
\texttt{comp\_bayes\_lm()} and has equivalent generic methods
implemented.

\hypertarget{casestudy}{%
\section{forestecology workflow: a case study}\label{casestudy}}

We present a case-study of the \texttt{forestecology} package's
functionality on data from the Smithsonian Conservation Biology
Institute (SCBI) large forest dynamics plot in Front Royal, VA, USA,
which is part of the ForestGEO global network of research sites
\citep[\citet{andersonteixeira_ctfs-forestgeo_2015}]{bourg_initial_2013}
\citep{bourg_initial_2013}. The 25.6 ha (640 x 400 m) plot is located at
the intersection of three of the major physiographic provinces of the
eastern US---the Blue Ridge, Ridge and Valley, and Piedmont
provinces---and is adjacent to the northern end of Shenandoah National
Park.

The \texttt{forestecology} package has the following ecological goals:
1) to evaluate the effect of competitor species identity using
permutation tests and 2) to evaluate model performance using spatial
cross-validation. To achieve these goals, we outline a basic analysis
sequence comprising of these four main steps:

\begin{enumerate}
\def\labelenumi{\arabic{enumi}.}
\tightlist
\item
  Compute the growth of stems based on two censuses.
\item
  Add spatial information:

  \begin{enumerate}
  \def\labelenumii{\arabic{enumii}.}
  \tightlist
  \item
    Define a buffer region of trees.
  \item
    Add spatial cross-validation block information.
  \end{enumerate}
\item
  Identify all focal trees and their competitors.
\item
  Apply model, which includes:

  \begin{enumerate}
  \def\labelenumii{\arabic{enumii}.}
  \tightlist
  \item
    Fit model.
  \item
    Compute fitted/predicted values.
  \item
    Visualize posterior distributions.
  \end{enumerate}
\end{enumerate}

We start by loading all necessary packages.

\begin{Shaded}
\begin{Highlighting}[]
\KeywordTok{library}\NormalTok{(tidyverse)}
\KeywordTok{library}\NormalTok{(lubridate)}
\KeywordTok{library}\NormalTok{(sf)}
\KeywordTok{library}\NormalTok{(patchwork)}
\KeywordTok{library}\NormalTok{(forestecology)}
\KeywordTok{library}\NormalTok{(blockCV)}

\CommentTok{# Resolve conflicting functions}
\NormalTok{filter <-}\StringTok{ }\NormalTok{dplyr}\OperatorTok{::}\NormalTok{filter}
\NormalTok{select <-}\StringTok{ }\NormalTok{dplyr}\OperatorTok{::}\NormalTok{select}
\end{Highlighting}
\end{Shaded}

\hypertarget{compute-growth}{%
\subsection{Step 1: Compute the growth of trees based on census
data}\label{compute-growth}}

The first step is to compute the growth of trees using data from two
censuses. \texttt{compute\_growth()} computes average annual growth
assuming census data that roughly follows ForestGEO standards. Despite
such standards, minor variations will still exist between sites, thereby
necessitating some data wrangling and checking. For example, the SCBI
site records all diameters at breast height (DBH) in millimeters
\citep{bourg_initial_2013}, whereas the Michigan Big Woods site records
them in centimeters \citep{allen_michigan_2020}.

We first load both 2008 and 2014 SCBI census data \texttt{.csv} files as
they existed on GitHub on November 20, 2020
\citep{gonzalez-akre_scbi-forestgeoscbi-forestgeo-data_2020} and perform
some data wrangling to both data sets. We then only consider a 9 ha
subsection of the 25.6 ha of the SCBI site, \texttt{gx} from 0--300
instead of 0--400 and \texttt{gy} from 300--600 instead of 0--640, in
order to speed up computation for purposes of this example.

\begin{Shaded}
\begin{Highlighting}[]
\NormalTok{census_}\DecValTok{2013}\NormalTok{_scbi <-}\StringTok{ }\KeywordTok{read_csv}\NormalTok{(}\StringTok{"scbi.stem2.csv"}\NormalTok{) }\OperatorTok
\StringTok{  }\KeywordTok{select}\NormalTok{(stemID, sp, }\DataTypeTok{date =}\NormalTok{ ExactDate, gx, gy, dbh, codes, status) }\OperatorTok
\StringTok{  }\KeywordTok{mutate}\NormalTok{(}
    \CommentTok{# Convert date from character to date}
    \DataTypeTok{date =} \KeywordTok{mdy}\NormalTok{(date),}
    \CommentTok{# Convert dbh to be in cm}
    \DataTypeTok{dbh =} \KeywordTok{as.numeric}\NormalTok{(dbh)}\OperatorTok{/}\DecValTok{10}
\NormalTok{  ) }\OperatorTok
\StringTok{  }\KeywordTok{filter}\NormalTok{(gx }\OperatorTok{<}\StringTok{ }\DecValTok{300}\NormalTok{, }\KeywordTok{between}\NormalTok{(gy, }\DecValTok{300}\NormalTok{, }\DecValTok{600}\NormalTok{))}

\NormalTok{census_}\DecValTok{2018}\NormalTok{_scbi <-}\StringTok{ }\KeywordTok{read_csv}\NormalTok{(}\StringTok{"scbi.stem3.csv"}\NormalTok{) }\OperatorTok
\StringTok{  }\KeywordTok{select}\NormalTok{(stemID, sp, }\DataTypeTok{date =}\NormalTok{ ExactDate, gx, gy, dbh, codes, status) }\OperatorTok
\StringTok{  }\KeywordTok{mutate}\NormalTok{(}
    \DataTypeTok{date =} \KeywordTok{mdy}\NormalTok{(date),}
    \DataTypeTok{dbh =} \KeywordTok{as.numeric}\NormalTok{(dbh)}\OperatorTok{/}\DecValTok{10}
\NormalTok{  ) }\OperatorTok
\StringTok{  }\KeywordTok{filter}\NormalTok{(gx }\OperatorTok{<}\StringTok{ }\DecValTok{300}\NormalTok{, }\KeywordTok{between}\NormalTok{(gy, }\DecValTok{300}\NormalTok{, }\DecValTok{600}\NormalTok{))}
\end{Highlighting}
\end{Shaded}

These two data frames are then supplied as arguments to
\texttt{compute\_growth()}, along with the \texttt{id} argument that
specifies the variable that uniquely identifies each tree-stem. Note
furthermore that we discard all resprouts in the later census (those
with \texttt{code\ ==\ R}), since we are only interested in the diameter
growth of surviving, and not resprouted, stems.

\begin{Shaded}
\begin{Highlighting}[]
\NormalTok{growth_scbi <-}
\StringTok{  }\KeywordTok{compute_growth}\NormalTok{(}
    \DataTypeTok{census_1 =}\NormalTok{ census_}\DecValTok{2013}\NormalTok{_scbi,}
    \DataTypeTok{census_2 =}\NormalTok{ census_}\DecValTok{2018}\NormalTok{_scbi }\OperatorTok\StringTok{ }\KeywordTok{filter}\NormalTok{(}\OperatorTok{!}\KeywordTok{str_detect}\NormalTok{(codes, }\StringTok{"R"}\NormalTok{)),}
    \DataTypeTok{id =} \StringTok{"stemID"}
\NormalTok{  )}
\NormalTok{growth_scbi}
\CommentTok{## Simple feature collection with 7954 features and 8 fields}
\CommentTok{## geometry type:  POINT}
\CommentTok{## dimension:      XY}
\CommentTok{## bbox:           xmin: 0.2 ymin: 300 xmax: 300 ymax: 600}
\CommentTok{## CRS:            NA}
\CommentTok{## # A tibble: 7,954 x 9}
\CommentTok{##   stemID sp     dbh1 codes1 status  dbh2 codes2 growth}
\CommentTok{##    <dbl> <fct> <dbl> <chr>  <chr>  <dbl> <chr>   <dbl>}
\CommentTok{## 1      4 nysy  13.6  M      A       14.2 M       0.103}
\CommentTok{## 2      5 havi   8.8  M      A        9.6 M;P     0.150}
\CommentTok{## 3      6 havi   3.25 NULL   A        4   M       0.140}
\CommentTok{## 4     77 qual  65.2  M      A       66   M       0.141}
\CommentTok{## 5     79 tiam  47.7  M      A       46.8 M      -0.161}
\CommentTok{## # ... with 7,949 more rows, and 1 more variable: geometry <POINT>}
\end{Highlighting}
\end{Shaded}

The output \texttt{growth\_scbi} is a single data frame of class
\texttt{sf} that includes variables \texttt{growth}, the average annual
growth in DBH (cm \(\cdot\) y\textsuperscript{-1}) for all stems that
were alive at both time points, and \texttt{geometry}, the \texttt{sf}
package's encoding of geolocations of type
\texttt{\textless{}POINT\textgreater{}}. In addition the species
variable \texttt{sp} is returned as a factor.\footnote{In our spatial
  cross-validation algorithm in Section \ref{spatial-cross-validation}
  issues can occur when rare species do not occur in the training set,
  but then are encountered in the test set. This risk is mitigated by
  representing \texttt{sp} as a factor variable, which has a complete
  list of all levels of the categorical variable.} Given that
\texttt{growth\_scbi} is of class \texttt{sf}, it can be easily plotted
in \texttt{ggplot2} using the \texttt{geom\_sf()} geometry as seen in
Figure \ref{fig:scbi-trees}.

\begin{Shaded}
\begin{Highlighting}[]
\KeywordTok{ggplot}\NormalTok{() }\OperatorTok{+}
\StringTok{  }\KeywordTok{geom_sf}\NormalTok{(}\DataTypeTok{data =}\NormalTok{ growth_scbi }\OperatorTok\StringTok{ }\KeywordTok{sample_n}\NormalTok{(}\DecValTok{500}\NormalTok{), }\KeywordTok{aes}\NormalTok{(}\DataTypeTok{size =}\NormalTok{ growth)) }\OperatorTok{+}\StringTok{ }
\StringTok{  }\KeywordTok{scale_size_binned}\NormalTok{(}\DataTypeTok{limits =} \KeywordTok{c}\NormalTok{(}\FloatTok{0.1}\NormalTok{, }\DecValTok{1}\NormalTok{))}
\end{Highlighting}
\end{Shaded}

\begin{figure}

{\centering \includegraphics[width=0.66\linewidth]{Figures/scbi-trees-1} 

}

\caption{Compute growth of trees based on census data: Map with growth of a random sample of 500 trees from a 9 ha subsection of the Smithsonian Conservation Biology Institute (SCBI) forest plot.}\label{fig:scbi-trees}
\end{figure}

\hypertarget{spatial-information}{%
\subsection{Step 2: Add spatial information}\label{spatial-information}}

The next step is to add additional spatial information to
\texttt{growth\_scbi}. The first element we add is a ``buffer region''
to the periphery of the study region. Since some of our model's
explanatory variables are cumulative, we must ensure that all trees
being modeled are not biased to have different neighbor structures. This
is of concern for trees at the boundary of study regions, for which all
neighbors will not be included in the censused stems. In order to
account for such edge effects, only trees that are not part of this
buffer region, i.e.~are part of the interior of the study region, will
have their growth modeled \citep{waller_applied_2004}.

Our model of interspecific competition relies on a spatial definition of
who the competitor trees are for focal trees of interest: all trees
within a distance \texttt{comp\_dist} of a focal tree are considered its
competitors. In our case we set this value at 7.5m, a value informed by
other studies \citep[\citet{uriarte_spatially_2004},
\citet{canham_neighborhood_2006}]{canham_neighborhood_2004}. Using this
value along with a manually constructed \texttt{sf} object
representation of the study region's boundary, we apply the
\texttt{add\_buffer\_variable()} to \texttt{growth\_scbi} to add a
\texttt{buffer} boolean variable. All trees with \texttt{buffer} as
\texttt{FALSE} will be our focal trees whose growth will be modeled,
whereas those with \texttt{TRUE} will only be considered as competitor
trees.

\begin{Shaded}
\begin{Highlighting}[]
\CommentTok{# Define competitive distance range}
\NormalTok{comp_dist <-}\StringTok{ }\FloatTok{7.5}

\CommentTok{# Manually construct study region boundary}
\NormalTok{study_region_scbi <-}\StringTok{ }\KeywordTok{tibble}\NormalTok{(}
  \DataTypeTok{x =} \KeywordTok{c}\NormalTok{(}\DecValTok{0}\NormalTok{, }\DecValTok{300}\NormalTok{, }\DecValTok{300}\NormalTok{, }\DecValTok{0}\NormalTok{, }\DecValTok{0}\NormalTok{),}
  \DataTypeTok{y =} \KeywordTok{c}\NormalTok{(}\DecValTok{300}\NormalTok{, }\DecValTok{300}\NormalTok{, }\DecValTok{600}\NormalTok{, }\DecValTok{600}\NormalTok{, }\DecValTok{300}\NormalTok{)}
\NormalTok{) }\OperatorTok
\StringTok{  }\KeywordTok{sf_polygon}\NormalTok{()}

\NormalTok{growth_scbi <-}\StringTok{ }\NormalTok{growth_scbi }\OperatorTok
\StringTok{  }\KeywordTok{add_buffer_variable}\NormalTok{(}\DataTypeTok{size =}\NormalTok{ comp_dist, }\DataTypeTok{region =}\NormalTok{ study_region_scbi)}
\end{Highlighting}
\end{Shaded}

The second element of spatial information are blocks corresponding to
folds of a spatial cross-validation algorithm used to estimate
out-of-sample model error. Conventional cross-validation algorithms
assign observations to folds by randomly resampling individual
observations. However, many of these algorithms assume that the
observations are independent. In the case of forest census data,
observations exhibit spatial autocorrelation. We therefore incorporate
this spatial dependence into the cross-validation algorithm with our
spatial blocks of trees \citep[
\citet{pohjankukka_estimating_2017}]{roberts_cross-validation_2017}.

In the example below, we first manually define four folds that partition
the study region as an \texttt{sf} object. We then use the output of the
\texttt{spatialBlock()} function from the \texttt{blockCV} package to
associate each tree in \texttt{growth\_scbi} to the correct fold
\texttt{foldID} \citep{valavi_blockcv_2019}. \footnote{In the Supporting
  Information we present an example where the folds themselves are also
  created automatically using \texttt{spatialBlock()} given a specified
  \texttt{cv\_block\_size}, as opposed to manually as in the example.}
Figure \ref{fig:scbi-spatial-information} illustrates the net effect of
adding these two elements of spatial information to
\texttt{growth\_scbi}.

\begin{Shaded}
\begin{Highlighting}[]
\CommentTok{# Manually define spatial blocks to act as folds}
\NormalTok{n_fold <-}\StringTok{ }\DecValTok{4}
\NormalTok{fold1 <-}\StringTok{ }\KeywordTok{rbind}\NormalTok{(}\KeywordTok{c}\NormalTok{(}\DecValTok{0}\NormalTok{, }\DecValTok{300}\NormalTok{), }\KeywordTok{c}\NormalTok{(}\DecValTok{150}\NormalTok{, }\DecValTok{300}\NormalTok{), }\KeywordTok{c}\NormalTok{(}\DecValTok{150}\NormalTok{, }\DecValTok{450}\NormalTok{), }\KeywordTok{c}\NormalTok{(}\DecValTok{0}\NormalTok{, }\DecValTok{450}\NormalTok{))}
\NormalTok{fold2 <-}\StringTok{ }\KeywordTok{rbind}\NormalTok{(}\KeywordTok{c}\NormalTok{(}\DecValTok{150}\NormalTok{, }\DecValTok{300}\NormalTok{), }\KeywordTok{c}\NormalTok{(}\DecValTok{300}\NormalTok{, }\DecValTok{300}\NormalTok{), }\KeywordTok{c}\NormalTok{(}\DecValTok{300}\NormalTok{, }\DecValTok{450}\NormalTok{), }\KeywordTok{c}\NormalTok{(}\DecValTok{150}\NormalTok{, }\DecValTok{450}\NormalTok{))}
\NormalTok{fold3 <-}\StringTok{ }\KeywordTok{rbind}\NormalTok{(}\KeywordTok{c}\NormalTok{(}\DecValTok{0}\NormalTok{, }\DecValTok{450}\NormalTok{), }\KeywordTok{c}\NormalTok{(}\DecValTok{150}\NormalTok{, }\DecValTok{450}\NormalTok{), }\KeywordTok{c}\NormalTok{(}\DecValTok{150}\NormalTok{, }\DecValTok{600}\NormalTok{), }\KeywordTok{c}\NormalTok{(}\DecValTok{0}\NormalTok{, }\DecValTok{600}\NormalTok{))}
\NormalTok{fold4 <-}\StringTok{ }\KeywordTok{rbind}\NormalTok{(}\KeywordTok{c}\NormalTok{(}\DecValTok{150}\NormalTok{, }\DecValTok{450}\NormalTok{), }\KeywordTok{c}\NormalTok{(}\DecValTok{300}\NormalTok{, }\DecValTok{450}\NormalTok{), }\KeywordTok{c}\NormalTok{(}\DecValTok{300}\NormalTok{, }\DecValTok{600}\NormalTok{), }\KeywordTok{c}\NormalTok{(}\DecValTok{150}\NormalTok{, }\DecValTok{600}\NormalTok{))}

\NormalTok{blocks_scbi <-}\StringTok{ }\KeywordTok{bind_rows}\NormalTok{(}
  \KeywordTok{sf_polygon}\NormalTok{(fold1), }\KeywordTok{sf_polygon}\NormalTok{(fold2), }\KeywordTok{sf_polygon}\NormalTok{(fold3), }
  \KeywordTok{sf_polygon}\NormalTok{(fold4)}
\NormalTok{) }\OperatorTok
\StringTok{  }\KeywordTok{mutate}\NormalTok{(}\DataTypeTok{folds =} \KeywordTok{c}\NormalTok{(}\DecValTok{1}\OperatorTok{:}\NormalTok{n_fold) }\OperatorTok\StringTok{ }\KeywordTok{factor}\NormalTok{())}

\CommentTok{# Associate each observation to a fold}
\NormalTok{spatial_block_scbi <-}\StringTok{ }\KeywordTok{spatialBlock}\NormalTok{(}
  \DataTypeTok{speciesData =}\NormalTok{ growth_scbi, }\DataTypeTok{k =}\NormalTok{ n_fold, }\DataTypeTok{selection =} \StringTok{"systematic"}\NormalTok{, }
  \DataTypeTok{blocks =}\NormalTok{ blocks_scbi, }\DataTypeTok{showBlocks =} \OtherTok{FALSE}\NormalTok{, }\DataTypeTok{verbose =} \OtherTok{FALSE}
\NormalTok{)}

\NormalTok{growth_scbi <-}\StringTok{ }\NormalTok{growth_scbi }\OperatorTok
\StringTok{  }\KeywordTok{mutate}\NormalTok{(}\DataTypeTok{foldID =}\NormalTok{ spatial_block_scbi}\OperatorTok{$}\NormalTok{foldID }\OperatorTok\StringTok{ }\KeywordTok{factor}\NormalTok{())}
\end{Highlighting}
\end{Shaded}

\begin{Shaded}
\begin{Highlighting}[]
\KeywordTok{ggplot}\NormalTok{() }\OperatorTok{+}
\StringTok{  }\KeywordTok{geom_sf}\NormalTok{(}\DataTypeTok{data =}\NormalTok{ blocks_scbi, }\DataTypeTok{fill =} \StringTok{"transparent"}\NormalTok{, }\DataTypeTok{linetype =} \StringTok{"dashed"}\NormalTok{) }\OperatorTok{+}
\StringTok{  }\KeywordTok{geom_sf_text}\NormalTok{(}\DataTypeTok{data =}\NormalTok{ growth_scbi }\OperatorTok\StringTok{ }\KeywordTok{sample_n}\NormalTok{(}\DecValTok{1000}\NormalTok{), }
               \KeywordTok{aes}\NormalTok{(}\DataTypeTok{label =}\NormalTok{ foldID, }\DataTypeTok{col =}\NormalTok{ buffer))}
\end{Highlighting}
\end{Shaded}

\begin{figure}

{\centering \includegraphics[width=0.66\linewidth]{Figures/scbi-spatial-information-1} 

}

\caption{Add spatial information: Buffer region and spatial cross-validation blocks (1 through 4). The location of each tree is marked with an integer indicating its fold, with folds delineated with solid lines. The color of each digit indicates whether the tree is part of the buffer region (and thus will only be considered as a competitor tree in our model) or is part of the interior of the study region (and thus is a focal tree whose growth is of modeled interest).}\label{fig:scbi-spatial-information}
\end{figure}

\hypertarget{focal-vs-comp}{%
\subsection{Step 3: Identify all focal and corresponding competitor
trees}\label{focal-vs-comp}}

The next step is to identify all focal trees and their corresponding
competitor trees. More specifically, identify all trees that are not
part of the buffer region, have a valid \texttt{growth} measurement, and
have at least one neighbor within 7.5m.
\texttt{create\_focal\_vs\_comp()} returns a new data frame of type
\texttt{sf}. On top of previously detailed arguments \texttt{comp\_dist}
and \texttt{id}, \texttt{create\_focal\_vs\_comp()} also requires an
\texttt{sf} object representation of the spatial cross-validation
blocks/folds as seen in Section \ref{spatial-information}.

\begin{Shaded}
\begin{Highlighting}[]
\NormalTok{focal_vs_comp_scbi <-}\StringTok{ }\NormalTok{growth_scbi }\OperatorTok
\StringTok{  }\KeywordTok{create_focal_vs_comp}\NormalTok{(comp_dist, }\DataTypeTok{blocks =}\NormalTok{ blocks_scbi, }\DataTypeTok{id =} \StringTok{"stemID"}\NormalTok{)}
\NormalTok{focal_vs_comp_scbi }\OperatorTok\StringTok{ }
\StringTok{  }\KeywordTok{select}\NormalTok{(focal_ID, focal_sp, geometry, growth, comp)}
\CommentTok{## # A tibble: 6,296 x 5}
\CommentTok{##   focal_ID focal_sp   geometry growth comp             }
\CommentTok{##      <dbl> <fct>       <POINT>  <dbl> <list>           }
\CommentTok{## 1        4 nysy     (14.2 428)  0.103 <tibble [20 x 4]>}
\CommentTok{## 2        5 havi      (9.4 436)  0.150 <tibble [32 x 4]>}
\CommentTok{## 3       79 tiam       (40 381) -0.161 <tibble [20 x 4]>}
\CommentTok{## 4       80 caca     (38.7 422)  0.253 <tibble [12 x 4]>}
\CommentTok{## 5       96 libe       (60 310)  0.262 <tibble [14 x 4]>}
\CommentTok{## # ... with 6,291 more rows}
\end{Highlighting}
\end{Shaded}

The resulting \texttt{focal\_vs\_comp\_scbi} has 6296 rows, representing
the subset of the 7954 trees in \texttt{growth\_scbi} that will be
considered as focal trees. Two new variables \texttt{focal\_ID} and
\texttt{focal\_sp} relate to tree-stem identification and species
information. Most notably however is a new variable \texttt{comp} which
contains information on all competitor trees saved in \texttt{tidyr}
package list-column format \citep{tidyr_package}. We flatten the
\texttt{comp} list-column for the tree with \texttt{focal\_ID} 4 in the
first row, here a \texttt{tibble\ {[}20\ ×\ 4{]}}, into regular columns
using \texttt{unnest()} from the \texttt{tidyr} package.

\begin{Shaded}
\begin{Highlighting}[]
\NormalTok{focal_vs_comp_scbi }\OperatorTok\StringTok{ }
\StringTok{  }\KeywordTok{filter}\NormalTok{(focal_ID }\OperatorTok{==}\StringTok{ }\DecValTok{4}\NormalTok{) }\OperatorTok\StringTok{ }
\StringTok{  }\KeywordTok{select}\NormalTok{(focal_ID, dbh, comp) }\OperatorTok\StringTok{ }
\StringTok{  }\KeywordTok{unnest}\NormalTok{(}\DataTypeTok{cols =} \StringTok{"comp"}\NormalTok{)}
\CommentTok{## # A tibble: 20 x 6}
\CommentTok{##   focal_ID   dbh comp_ID  dist comp_sp comp_basal_area}
\CommentTok{##      <dbl> <dbl>   <dbl> <dbl> <fct>             <dbl>}
\CommentTok{## 1        4  13.6    1836  7.48 tiam            0.0176 }
\CommentTok{## 2        4  13.6    1847  2.81 nysy            0.00332}
\CommentTok{## 3        4  13.6    1848  1.62 nysy            0.00396}
\CommentTok{## 4        4  13.6    1849  2.62 nysy            0.00535}
\CommentTok{## 5        4  13.6    1850  2.98 havi            0.00472}
\CommentTok{## # ... with 15 more rows}
\end{Highlighting}
\end{Shaded}

We observe that for this focal tree, we have 4 variables of information
on its 20 competitor trees: their unique tree-stem ID number, their
distance to the focal tree (all \(\leq\) 7.5), their species, and their
basal area (in m\(^2\)) calculated as
\(\frac{\pi \times (\text{DBH/2})^2}{10000}\) where \(DBH\) is the value
from the earlier of the two censuses in cm. Saving our focal versus
competitor information in list-column minimizes redundancy since we do
not repeat information on the focal tree 20 times. The spatial
distribution of these trees is visualized in Figure
\ref{fig:scbi-focal-vs-comp-map}.

\begin{figure}

{\centering \includegraphics[width=0.66\linewidth]{Figures/scbi-focal-vs-comp-map-1} 

}

\caption{Identify all focal and corresponding competitor trees: The dashed circle extends 7.5m away from the focal tree 4 while all 20 competitor trees are within this circle.}\label{fig:scbi-focal-vs-comp-map}
\end{figure}

\hypertarget{model-fit-predict}{%
\subsection{Step 4: Fit model}\label{model-fit-predict}}

The final step is to fit a model for the growth of all focal trees. We
fit the competition Bayesian linear regression model outlined in
Equation \ref{eq:model} using \texttt{comp\_bayes\_lm()}, which has an
option to specify prior distributions on all parameters of interest
(chosen to be the defaults specified in \texttt{?comp\_bayes\_lm}).

\begin{Shaded}
\begin{Highlighting}[]
\NormalTok{comp_bayes_lm_scbi <-}\StringTok{ }\NormalTok{focal_vs_comp_scbi }\OperatorTok
\StringTok{  }\KeywordTok{comp_bayes_lm}\NormalTok{(}\DataTypeTok{prior_param =} \OtherTok{NULL}\NormalTok{)}
\end{Highlighting}
\end{Shaded}

The returned \texttt{comp\_bayes\_lm\_scbi} output is an object of S3
class type \texttt{comp\_bayes\_lm} containing the posterior values of
all parameters in our competition Bayesian linear regression. This class
of object includes three generic methods. First, the generic for
\texttt{print()} displays the names of all prior \& posterior parameters
along with the model formula:

\begin{Shaded}
\begin{Highlighting}[]
\NormalTok{comp_bayes_lm_scbi}
\CommentTok{## Bayesian linear regression model parameters with a multivariate Normal}
\CommentTok{## likelihood. See ?comp_bayes_lm for details:}
\CommentTok{## }
\CommentTok{##   parameter_type           prior posterior}
\CommentTok{## 1 Inverse-Gamma on sigma^2 a_0   a_star   }
\CommentTok{## 2 Inverse-Gamma on sigma^2 b_0   b_star   }
\CommentTok{## 3 Multivariate t on beta   mu_0  mu_star  }
\CommentTok{## 4 Multivariate t on beta   V_0   V_star   }
\CommentTok{## }
\CommentTok{## Model formula:}
\CommentTok{## growth ~ sp + dbh + dbh * sp + acne * sp + acru * sp + amar * sp + astr}
\CommentTok{## * sp + caca * sp + caco * sp + cade * sp + cagl * sp + caovl * sp + cato}
\CommentTok{## * sp + ceca * sp + ceoc * sp + chvi * sp + cofl * sp + crpr * sp + crsp}
\CommentTok{## * sp + divi * sp + elum * sp + fagr * sp + fram * sp + frni * sp + frpe}
\CommentTok{## * sp + havi * sp + ilve * sp + juci * sp + juni * sp + libe * sp + litu}
\CommentTok{## * sp + nysy * sp + pist * sp + pivi * sp + ploc * sp + prav * sp + prse}
\CommentTok{## * sp + qual * sp + quco * sp + qufa * sp + qumi * sp + qupr * sp + quru}
\CommentTok{## * sp + quve * sp + rops * sp + saal * sp + saca * sp + tiam * sp + ulam}
\CommentTok{## * sp + ulru * sp + unk * sp + vipr * sp}
\end{Highlighting}
\end{Shaded}

Next, the generic for \texttt{predict()} takes the posterior parameter
values in \texttt{comp\_bayes\_lm\_scbi} and the predictor variables in
\texttt{newdata} and outputs a vector \texttt{growth\_hat} of
fitted/predicted values \(\widehat{y}\) of the DBH for each focal tree
computed from the posterior predictive distribution.

\begin{Shaded}
\begin{Highlighting}[]
\NormalTok{focal_vs_comp_scbi <-}\StringTok{ }\NormalTok{focal_vs_comp_scbi }\OperatorTok
\StringTok{  }\KeywordTok{mutate}\NormalTok{(}\DataTypeTok{growth_hat =} \KeywordTok{predict}\NormalTok{(comp_bayes_lm_scbi, }\DataTypeTok{newdata =}\NormalTok{ focal_vs_comp_scbi))}
\end{Highlighting}
\end{Shaded}

\begin{Shaded}
\begin{Highlighting}[]
\NormalTok{focal_vs_comp_scbi }\OperatorTok\StringTok{ }
\StringTok{  }\KeywordTok{select}\NormalTok{(focal_ID, focal_sp, dbh, growth, growth_hat)}
\CommentTok{## # A tibble: 6,296 x 5}
\CommentTok{##   focal_ID focal_sp   dbh growth growth_hat}
\CommentTok{##      <dbl> <fct>    <dbl>  <dbl>      <dbl>}
\CommentTok{## 1        4 nysy     13.6   0.103     0.0809}
\CommentTok{## 2        5 havi      8.8   0.150     0.112 }
\CommentTok{## 3       79 tiam     47.7  -0.161     0.229 }
\CommentTok{## 4       80 caca      5.15  0.253     0.121 }
\CommentTok{## 5       96 libe      2.3   0.262     0.142 }
\CommentTok{## # ... with 6,291 more rows}
\end{Highlighting}
\end{Shaded}

We then compare the observed and fitted/predicted growths to compute the
root mean squared error (RMSE) of our model fit.

\begin{Shaded}
\begin{Highlighting}[]
\NormalTok{model_rmse <-}\StringTok{ }\NormalTok{focal_vs_comp_scbi }\OperatorTok
\StringTok{  }\KeywordTok{rmse}\NormalTok{(}\DataTypeTok{truth =}\NormalTok{ growth, }\DataTypeTok{estimate =}\NormalTok{ growth_hat) }\OperatorTok
\StringTok{  }\KeywordTok{pull}\NormalTok{(.estimate)}
\NormalTok{model_rmse}
\CommentTok{## [1] 0.128}
\end{Highlighting}
\end{Shaded}

Lastly, the generic for \texttt{ggplot2::autoplot()} allows us to plot
the posterior distribution of all parameters in Figure
\ref{fig:scbi-posterior-viz}.

\begin{Shaded}
\begin{Highlighting}[]
\CommentTok{# Plot posteriors for only a subset of species}
\NormalTok{sp_to_plot <-}\StringTok{ }\KeywordTok{c}\NormalTok{(}\StringTok{"litu"}\NormalTok{, }\StringTok{"quru"}\NormalTok{, }\StringTok{"cagl"}\NormalTok{)}

\NormalTok{plot1 <-}\StringTok{ }\KeywordTok{autoplot}\NormalTok{(comp_bayes_lm_scbi, }\DataTypeTok{type =} \StringTok{"intercepts"}\NormalTok{, }
                  \DataTypeTok{sp_to_plot =}\NormalTok{ sp_to_plot)}
\NormalTok{plot2 <-}\StringTok{ }\KeywordTok{autoplot}\NormalTok{(comp_bayes_lm_scbi, }\DataTypeTok{type =} \StringTok{"dbh_slopes"}\NormalTok{, }
                  \DataTypeTok{sp_to_plot =}\NormalTok{ sp_to_plot)}
\NormalTok{plot3 <-}\StringTok{ }\KeywordTok{autoplot}\NormalTok{(comp_bayes_lm_scbi, }\DataTypeTok{type =} \StringTok{"competition"}\NormalTok{, }
                  \DataTypeTok{sp_to_plot =}\NormalTok{ sp_to_plot)}

\CommentTok{# Combine plots using patchwork}
\NormalTok{(plot1 }\OperatorTok{|}\StringTok{ }\NormalTok{plot2) }\OperatorTok{/}\StringTok{ }\NormalTok{plot3}
\end{Highlighting}
\end{Shaded}

\begin{figure}

{\centering \includegraphics[width=1\linewidth]{Figures/scbi-posterior-viz-1} 

}

\caption{Fit model: Posterior distributions of all parameters. For compactness we include only three species.}\label{fig:scbi-posterior-viz}
\end{figure}

These plots visualize the posterior distributions of parameters from
Equation \ref{eq:model}. For many package users they will be of interest
because they give insight into the species-specific competitive
interactions. Setting \texttt{type\ =\ "intercepts"} returns
species-specific posterior distributions for \(\beta_{0,j}\) and
\texttt{type\ =\ "dbh\_slopes"} for \(\beta_{dbh,j}\). Setting
\texttt{type\ =\ "competition"} returns competition coefficients
\(\lambda_{j,k}\) where negative values indicate a competitor species
which slows the growth of a focal species. Here, for example, we see
that \texttt{litu} tulip poplars have a strong negative effect on the
growth of conspecifics but relatively lesser effect on neighbors of the
other two species.

Currently the \texttt{forestecology} package can only fit the
competition Bayesian linear regression model outlined in Equation
\ref{eq:model}. However, it can be extended to any model implemented in
a function similar to \texttt{comp\_bayes\_lm()} that uses data frames
of similar format to \texttt{focal\_vs\_comp} as input.

\hypertarget{evaluate-the-effect-of-competitor-species-identity-using-permutation-tests}{%
\subsection{Evaluate the effect of competitor species identity using
permutation
tests}\label{evaluate-the-effect-of-competitor-species-identity-using-permutation-tests}}

To evaluate the effect of competitor species identity, we use the four
steps of our analysis sequence answer along with a permutation test:
Under a null hypothesis where competitor species identity does not
matter, we permute/shuffle this variable within each focal tree, compute
the RMSE (the test statistic of interest), repeat this process several
times to construct a null distribution of the RMSE, and compare it to
the observed RMSE to assess significance. Going back to our example in
Section \ref{focal-vs-comp} of focal tree with \texttt{focal\_ID} 4 and
its 20 competitors, the permutation test randomly resamples only the
\texttt{comp\_sp} variable with replacement, leaving all other variables
intact. The resampling with replacement is nested within each focal tree
in order to preserve neighborhood structure. We once again use
\texttt{comp\_bayes\_lm()} as in Section \ref{model-fit-predict}, but
with \texttt{run\_shuffle\ =\ TRUE}.

\begin{Shaded}
\begin{Highlighting}[]
\NormalTok{comp_bayes_lm_scbi_shuffle <-}\StringTok{ }\NormalTok{focal_vs_comp_scbi }\OperatorTok
\StringTok{  }\KeywordTok{comp_bayes_lm}\NormalTok{(}\DataTypeTok{prior_param =} \OtherTok{NULL}\NormalTok{, }\DataTypeTok{run_shuffle =} \OtherTok{TRUE}\NormalTok{)}

\NormalTok{focal_vs_comp_scbi <-}\StringTok{ }\NormalTok{focal_vs_comp_scbi }\OperatorTok
\StringTok{  }\KeywordTok{mutate}\NormalTok{(}
    \DataTypeTok{growth_hat_shuffle =} \KeywordTok{predict}\NormalTok{(comp_bayes_lm_scbi_shuffle, }
                                 \DataTypeTok{newdata =}\NormalTok{ focal_vs_comp_scbi)}
\NormalTok{  )}
\end{Highlighting}
\end{Shaded}

\begin{Shaded}
\begin{Highlighting}[]
\NormalTok{model_rmse_shuffle <-}\StringTok{ }\NormalTok{focal_vs_comp_scbi }\OperatorTok
\StringTok{  }\KeywordTok{rmse}\NormalTok{(}\DataTypeTok{truth =}\NormalTok{ growth, }\DataTypeTok{estimate =}\NormalTok{ growth_hat_shuffle) }\OperatorTok
\StringTok{  }\KeywordTok{pull}\NormalTok{(.estimate)}
\NormalTok{model_rmse_shuffle}
\CommentTok{## [1] 0.131}
\end{Highlighting}
\end{Shaded}

The resulting RMSE of 0.131 based on the permutation test is larger than
the earlier RMSE of 0.128, suggesting that models that do incorporate
competitor species identity better fit the data.

\hypertarget{spatial-cross-validation}{%
\subsection{Evaluate model performance using spatial
cross-validation}\label{spatial-cross-validation}}

We answer the second of our two questions: how can we obtain an accurate
estimate of model performance/error? The model fits and predictions in
Section \ref{model-fit-predict} all suffer from a common failing: they
use the same data to both fit the model and to assess the model's
performance using the RMSE. As argued by
\citet{roberts_cross-validation_2017}, this can lead to overly
optimistic assessments of model quality as the models can be overfit, in
particular in situations where spatial-autocorrelation is present. To
mitigate the effects of such overfitting, we use a spatially block
cross-validation algorithm.

To this end, we use the \texttt{foldID} variable defined in Section
\ref{spatial-information} whereby all focal trees are assigned to one of
4 spatially contiguous blocks that act as folds in our cross-validation
routine. Figure \ref{fig:scbi-spatial-cross-validation-schematic}
presents a schematic illustrating this scheme. We fit the model to all
focal trees in the training set, apply the model to all focal trees in
the test set to compute fitted/predicted values, and compute the RMSE of
the observed versus predicted growths. We repeat this procedure 3 more
times with each of the three remaining folds acting as the test set and
then average all four resulting RMSE's. Furthermore, in order to
maintain spatial independence between the test and training set, a
``fold buffer'' that extend outwards from the boundary of the test set
is computed; all trees falling within this fold buffer are excluded from
the training set.

\begin{figure}

{\centering \includegraphics[width=0.66\linewidth]{Figures/scbi-spatial-cross-validation-schematic-1} 

}

\caption{Schematic of spatial cross-validation: Using the k = 1 fold (bottom-left) as the test set, k = 2 through 4 as the training set, along with a "fold buffer."}\label{fig:scbi-spatial-cross-validation-schematic}
\end{figure}

This algorithm is implemented in \texttt{run\_cv()}, which is a wrapper
function to both \texttt{comp\_bayes\_lm()} that fits the model and
\texttt{predict()} that returns fitted/predicted values. We once again
compute the RMSE.

\begin{Shaded}
\begin{Highlighting}[]
\NormalTok{focal_vs_comp_scbi <-}\StringTok{ }\NormalTok{focal_vs_comp_scbi }\OperatorTok
\StringTok{  }\KeywordTok{run_cv}\NormalTok{(}\DataTypeTok{comp_dist =}\NormalTok{ comp_dist, }\DataTypeTok{blocks =}\NormalTok{ blocks_scbi)}
\end{Highlighting}
\end{Shaded}

\begin{Shaded}
\begin{Highlighting}[]
\NormalTok{model_rmse_cv <-}\StringTok{ }\NormalTok{focal_vs_comp_scbi }\OperatorTok
\StringTok{  }\KeywordTok{rmse}\NormalTok{(}\DataTypeTok{truth =}\NormalTok{ growth, }\DataTypeTok{estimate =}\NormalTok{ growth_hat) }\OperatorTok
\StringTok{  }\KeywordTok{pull}\NormalTok{(.estimate)}
\NormalTok{model_rmse_cv}
\CommentTok{## [1] 0.14}
\end{Highlighting}
\end{Shaded}

The resulting RMSE of 0.14 computed using cross-validation is larger
than the earlier RMSE of 0.128, suggesting that models that do not take
the inherent spatial autocorrelation of the data into account generate
error estimates that are overly optimistic; in our case RMSE's that are
too low.

\hypertarget{importance-of-spatial-cross-validation}{%
\section{Importance of spatial
cross-validation}\label{importance-of-spatial-cross-validation}}

\texttt{run\_cv()} also accepts the \texttt{run\_shuffle} argument. This
permutes the competitor species, as described above, but does so when
calculating predicted growth with the cross validated scheme. Figure
\ref{fig:scbi-simulation} compares model performance when permuting
competitor species and calculating RMSE with and without
cross-validation. Without cross-validation the competitor identity did
matter, the non-permuted competitor species had a much lower RMSE than
the permuted one. But once we include the spatial cross-validation, this
improvement disappears. These results suggest that in this 9 ha subplot
of the SCBI plot competitive interactions do not depend on the identity
of the competitor, which is the opposite of what has been observed in
other locations (\citet{allen_permutation_2020}
\citet{uriarte_spatially_2004}). This highlights the importance of
cross-validation, without it the model was overfit.

\begin{figure}

{\centering \includegraphics[width=1\linewidth]{simulation_results/2021-03-03_scbi_49_shuffles} 

}

\caption{Root mean squared error of models for standard, permuted, and spatial cross-validated error estimates. The dotted lines show non-permuted competitor identity, while the histogras so the RMSE for 49 permutations. The colors indicate whether cross validaton was used.}\label{fig:scbi-simulation}
\end{figure}

\hypertarget{conclusion}{%
\section{Conclusion}\label{conclusion}}

The \texttt{forestecology} package provides an accessible way to fit and
test models of neighborhood competition. While the package is designed
with ForestGEO plot data in mind, we envision that it can be modified to
work on i) any single large, mapped forest plot in which at least two
measurements of each individual have been taken, e.g.~the US Forest
Service Forest Inventory and Analysis plots or ii) more generally to
model interactions of any community of mapped sessile organisms
\citep{smith_forest_2002}. In future versions of \texttt{forestecology}
we also hope to make it possible to model plant mortality in addition to
plant growth. The package follows the guidelines for \texttt{tidy} data,
leverages the \texttt{sf} package for spatial data, and S3 open-oriented
model structure. We hope that the package will increase the use of
neighborhood competition models to better understand what structures
plant competition.

\hypertarget{acknowledgments}{%
\section{Acknowledgments}\label{acknowledgments}}

The authors thank Sophie Li for their feedback on the package interface.
The authors declare no conflicts of interest.

\hypertarget{authors-contributions}{%
\section{Author's contributions}\label{authors-contributions}}

AYK and DNA conceived the ideas and coded a draft of the package. AYK
wrote an initial manuscript draft. SPC rewrote much of the package's
code to align with R and ``tidy'' best practices
\citep{wickham_welcome_2019}. All authors contributed to subsequent
drafts and gave final approval for manuscript.

\hypertarget{data-accessibility}{%
\section{Data accessibility}\label{data-accessibility}}

We intend to archive all data and source code for this manuscript on
GitHub at \url{https://github.com/rudeboybert/forestecology}. This
repository will be archived on Zenodo upon acceptance. The example
Smithsonian Conservation Biology Institute census data used are
available on GitHub at
\url{https://github.com/SCBI-ForestGEO/SCBI-ForestGEO-Data/tree/master/tree_main_census/data/census-csv-files}
and are archived on Zenodo at
\url{https://doi.org/10.5281/zenodo.2649301}
\citep{gonzalez-akre_scbi-forestgeoscbi-forestgeo-data_2020}.

\bibliographystyle{agsm}
\bibliography{paper.bib}

\end{document}
